% Options for packages loaded elsewhere
\PassOptionsToPackage{unicode}{hyperref}
\PassOptionsToPackage{hyphens}{url}
%
\documentclass[
]{book}
\usepackage{amsmath,amssymb}
\usepackage{iftex}
\ifPDFTeX
  \usepackage[T1]{fontenc}
  \usepackage[utf8]{inputenc}
  \usepackage{textcomp} % provide euro and other symbols
\else % if luatex or xetex
  \usepackage{unicode-math} % this also loads fontspec
  \defaultfontfeatures{Scale=MatchLowercase}
  \defaultfontfeatures[\rmfamily]{Ligatures=TeX,Scale=1}
\fi
\usepackage{lmodern}
\ifPDFTeX\else
  % xetex/luatex font selection
\fi
% Use upquote if available, for straight quotes in verbatim environments
\IfFileExists{upquote.sty}{\usepackage{upquote}}{}
\IfFileExists{microtype.sty}{% use microtype if available
  \usepackage[]{microtype}
  \UseMicrotypeSet[protrusion]{basicmath} % disable protrusion for tt fonts
}{}
\makeatletter
\@ifundefined{KOMAClassName}{% if non-KOMA class
  \IfFileExists{parskip.sty}{%
    \usepackage{parskip}
  }{% else
    \setlength{\parindent}{0pt}
    \setlength{\parskip}{6pt plus 2pt minus 1pt}}
}{% if KOMA class
  \KOMAoptions{parskip=half}}
\makeatother
\usepackage{xcolor}
\usepackage{color}
\usepackage{fancyvrb}
\newcommand{\VerbBar}{|}
\newcommand{\VERB}{\Verb[commandchars=\\\{\}]}
\DefineVerbatimEnvironment{Highlighting}{Verbatim}{commandchars=\\\{\}}
% Add ',fontsize=\small' for more characters per line
\usepackage{framed}
\definecolor{shadecolor}{RGB}{248,248,248}
\newenvironment{Shaded}{\begin{snugshade}}{\end{snugshade}}
\newcommand{\AlertTok}[1]{\textcolor[rgb]{0.94,0.16,0.16}{#1}}
\newcommand{\AnnotationTok}[1]{\textcolor[rgb]{0.56,0.35,0.01}{\textbf{\textit{#1}}}}
\newcommand{\AttributeTok}[1]{\textcolor[rgb]{0.13,0.29,0.53}{#1}}
\newcommand{\BaseNTok}[1]{\textcolor[rgb]{0.00,0.00,0.81}{#1}}
\newcommand{\BuiltInTok}[1]{#1}
\newcommand{\CharTok}[1]{\textcolor[rgb]{0.31,0.60,0.02}{#1}}
\newcommand{\CommentTok}[1]{\textcolor[rgb]{0.56,0.35,0.01}{\textit{#1}}}
\newcommand{\CommentVarTok}[1]{\textcolor[rgb]{0.56,0.35,0.01}{\textbf{\textit{#1}}}}
\newcommand{\ConstantTok}[1]{\textcolor[rgb]{0.56,0.35,0.01}{#1}}
\newcommand{\ControlFlowTok}[1]{\textcolor[rgb]{0.13,0.29,0.53}{\textbf{#1}}}
\newcommand{\DataTypeTok}[1]{\textcolor[rgb]{0.13,0.29,0.53}{#1}}
\newcommand{\DecValTok}[1]{\textcolor[rgb]{0.00,0.00,0.81}{#1}}
\newcommand{\DocumentationTok}[1]{\textcolor[rgb]{0.56,0.35,0.01}{\textbf{\textit{#1}}}}
\newcommand{\ErrorTok}[1]{\textcolor[rgb]{0.64,0.00,0.00}{\textbf{#1}}}
\newcommand{\ExtensionTok}[1]{#1}
\newcommand{\FloatTok}[1]{\textcolor[rgb]{0.00,0.00,0.81}{#1}}
\newcommand{\FunctionTok}[1]{\textcolor[rgb]{0.13,0.29,0.53}{\textbf{#1}}}
\newcommand{\ImportTok}[1]{#1}
\newcommand{\InformationTok}[1]{\textcolor[rgb]{0.56,0.35,0.01}{\textbf{\textit{#1}}}}
\newcommand{\KeywordTok}[1]{\textcolor[rgb]{0.13,0.29,0.53}{\textbf{#1}}}
\newcommand{\NormalTok}[1]{#1}
\newcommand{\OperatorTok}[1]{\textcolor[rgb]{0.81,0.36,0.00}{\textbf{#1}}}
\newcommand{\OtherTok}[1]{\textcolor[rgb]{0.56,0.35,0.01}{#1}}
\newcommand{\PreprocessorTok}[1]{\textcolor[rgb]{0.56,0.35,0.01}{\textit{#1}}}
\newcommand{\RegionMarkerTok}[1]{#1}
\newcommand{\SpecialCharTok}[1]{\textcolor[rgb]{0.81,0.36,0.00}{\textbf{#1}}}
\newcommand{\SpecialStringTok}[1]{\textcolor[rgb]{0.31,0.60,0.02}{#1}}
\newcommand{\StringTok}[1]{\textcolor[rgb]{0.31,0.60,0.02}{#1}}
\newcommand{\VariableTok}[1]{\textcolor[rgb]{0.00,0.00,0.00}{#1}}
\newcommand{\VerbatimStringTok}[1]{\textcolor[rgb]{0.31,0.60,0.02}{#1}}
\newcommand{\WarningTok}[1]{\textcolor[rgb]{0.56,0.35,0.01}{\textbf{\textit{#1}}}}
\usepackage{longtable,booktabs,array}
\usepackage{calc} % for calculating minipage widths
% Correct order of tables after \paragraph or \subparagraph
\usepackage{etoolbox}
\makeatletter
\patchcmd\longtable{\par}{\if@noskipsec\mbox{}\fi\par}{}{}
\makeatother
% Allow footnotes in longtable head/foot
\IfFileExists{footnotehyper.sty}{\usepackage{footnotehyper}}{\usepackage{footnote}}
\makesavenoteenv{longtable}
\usepackage{graphicx}
\makeatletter
\def\maxwidth{\ifdim\Gin@nat@width>\linewidth\linewidth\else\Gin@nat@width\fi}
\def\maxheight{\ifdim\Gin@nat@height>\textheight\textheight\else\Gin@nat@height\fi}
\makeatother
% Scale images if necessary, so that they will not overflow the page
% margins by default, and it is still possible to overwrite the defaults
% using explicit options in \includegraphics[width, height, ...]{}
\setkeys{Gin}{width=\maxwidth,height=\maxheight,keepaspectratio}
% Set default figure placement to htbp
\makeatletter
\def\fps@figure{htbp}
\makeatother
\setlength{\emergencystretch}{3em} % prevent overfull lines
\providecommand{\tightlist}{%
  \setlength{\itemsep}{0pt}\setlength{\parskip}{0pt}}
\setcounter{secnumdepth}{5}
\usepackage{ctex}

\usepackage{booktabs}
\usepackage{amsthm,mathrsfs}
\usepackage{longtable}
\makeatletter
\def\thm@space@setup{%
  \thm@preskip=8pt plus 2pt minus 4pt
  \thm@postskip=\thm@preskip
}
\makeatother

\usepackage{geometry}
\geometry{
    left=2cm,   % 左边距
    right=2cm,  % 右边距
    top=2cm,    % 上边距
    bottom=1.5cm, % 下边距
}

% 设置中文字体
\setCJKmainfont[ItalicFont={楷体}, BoldFont={黑体}]{宋体} %衬线字体 缺省中文字体为
\setCJKsansfont{黑体} % 设置无衬线字体
\setCJKmonofont{仿宋} %中文等宽字体

% 设置英文字体
\usepackage{fontspec}
\setmainfont{Times New Roman} % 设置英文主要字体
\setsansfont{Arial} % 设置无衬线字体
\setmonofont{Lucida Console} % 设置等宽字体

% 设置字体大小
\usepackage[fontsize=13pt]{fontsize}
\ifLuaTeX
  \usepackage{selnolig}  % disable illegal ligatures
\fi
\usepackage[]{natbib}
\bibliographystyle{plainnat}
\IfFileExists{bookmark.sty}{\usepackage{bookmark}}{\usepackage{hyperref}}
\IfFileExists{xurl.sty}{\usepackage{xurl}}{} % add URL line breaks if available
\urlstyle{same}
\hypersetup{
  pdftitle={LZ 文档},
  pdfauthor={作者:荔枝 earth\_farmer@outlook.com},
  hidelinks,
  pdfcreator={LaTeX via pandoc}}

\title{LZ 文档}
\author{作者:荔枝 \href{mailto:earth_farmer@outlook.com}{\nolinkurl{earth\_farmer@outlook.com}}}
\date{更新:2023-12-17}

\begin{document}
\maketitle

{
\setcounter{tocdepth}{1}
\tableofcontents
}
\chapter*{前言}\label{ux524dux8a00}
\addcontentsline{toc}{chapter}{前言}

本包致力于简化生信分析流程和批量分析。目前主要为RNAseq分析流程,后期会加入多组学联合
分析流程。

\section*{为何阅读本书}\label{ux4e3aux4f55ux9605ux8bfbux672cux4e66}
\addcontentsline{toc}{section}{为何阅读本书}

本书是为LZ R package写的使用文档。旨在让完全没有编程基础或R基础的人学会使用LZ包来进
行一些生信分析。LZ包致力于简化生信分析流程和批量分析。
目前LZ包已经完成了RNAseq下游分析流程的大部分,后期完善后还有加入更多的功能。例如
RNAseq的上有分析流程,多组学联合分析流程,单细胞分析流程。
通过学习完本书,您将会在不需要系统学习R语言的情况下快速分析测序数据,如有疑问请发
email至\texttt{earth\_farmer@outlook.com},尽量有答复但不予保证。

\section*{本书结构}\label{ux672cux4e66ux7ed3ux6784}
\addcontentsline{toc}{section}{本书结构}

\section*{致谢}\label{ux81f4ux8c22}
\addcontentsline{toc}{section}{致谢}

\chapter{安装}\label{install}

\section{安装R及Rsudio环境}\label{ux5b89ux88c5rux53carsudioux73afux5883}

\textbf{LZ} R包可以从Github上安装。
先安装R及Rstudio, 前者是核心,后者是编辑器(写代码的地方)。
1. 安装\href{https://mirrors.tuna.tsinghua.edu.cn/CRAN/bin/}{R最新版},根据系统自行选择
版本,win用户可以直接点\href{https://mirrors.tuna.tsinghua.edu.cn/CRAN/bin/windows/base/R-4.3.2-win.exe}{R-4.3.2}下载。
2. 安装\href{https://posit.co/download/rstudio-desktop/}{Rstudio最新版},win用户可直接点\href{https://download1.rstudio.org/electron/windows/RStudio-2023.09.1-494.exe}{Rstudio Desktop}下载
3. Win电脑可以考虑安装R版本对应的\href{https://cran.r-project.org/bin/windows/Rtools/}{Rtools} (可选项,新手可以不安装)
4. \textbf{重要提示:请卸载或者至少退出一切杀毒软件(微软自带的不用退出),否则安装包时可能会出现难以解决的奇怪bug。}

\section{安装LZ包}\label{ux5b89ux88c5lzux5305}

\begin{enumerate}
\def\labelenumi{\arabic{enumi}.}
\tightlist
\item
  安装LZ包所需的依赖包
\end{enumerate}

\begin{Shaded}
\begin{Highlighting}[]
\CommentTok{\# 安装biocondutor}
\ControlFlowTok{if}\NormalTok{ (}\SpecialCharTok{!}\FunctionTok{require}\NormalTok{(}\StringTok{"BiocManager"}\NormalTok{, }\AttributeTok{quietly =} \ConstantTok{TRUE}\NormalTok{)) \{}
  \FunctionTok{install.packages}\NormalTok{(}\StringTok{"BiocManager"}\NormalTok{)}
\NormalTok{  BiocManager}\SpecialCharTok{::}\FunctionTok{install}\NormalTok{(}\AttributeTok{version =} \StringTok{"3.18"}\NormalTok{) }\CommentTok{\# 4.3 == 3.18}
\NormalTok{\}}

\CommentTok{\# 设置镜像(可选)}
\FunctionTok{options}\NormalTok{(}\StringTok{"repos"} \OtherTok{=} \FunctionTok{c}\NormalTok{(}\AttributeTok{CRAN =} \StringTok{"https://mirrors.tuna.tsinghua.edu.cn/CRAN/"}\NormalTok{)) }
\FunctionTok{options}\NormalTok{(}\AttributeTok{BioC\_mirror=}\StringTok{"https://mirrors.tuna.tsinghua.edu.cn/bioconductor"}\NormalTok{)}

\CommentTok{\# 安装LZ依赖包}
\CommentTok{\# 安装cran包}
\NormalTok{cran\_pack }\OtherTok{\textless{}{-}} \FunctionTok{c}\NormalTok{(}\StringTok{\textquotesingle{}devtools\textquotesingle{}}\NormalTok{, }\StringTok{\textquotesingle{}prettydoc\textquotesingle{}}\NormalTok{, }\StringTok{\textquotesingle{}Hmisc\textquotesingle{}}\NormalTok{, }\StringTok{\textquotesingle{}markdown\textquotesingle{}}\NormalTok{, }
               \StringTok{\textquotesingle{}Hmisc\textquotesingle{}}\NormalTok{, }\StringTok{\textquotesingle{}tidyverse\textquotesingle{}}\NormalTok{)}
\ControlFlowTok{for}\NormalTok{ (p }\ControlFlowTok{in}\NormalTok{ cran\_pack) \{ }
  \ControlFlowTok{if}\NormalTok{ (}\SpecialCharTok{!}\FunctionTok{requireNamespace}\NormalTok{(p, }\AttributeTok{quietly =}\NormalTok{ T)) }\FunctionTok{install.packages}\NormalTok{(p) }
\NormalTok{\}}
\CommentTok{\# 安装bioconductor包}
\NormalTok{bioc\_pack }\OtherTok{\textless{}{-}} \FunctionTok{c}\NormalTok{(}\StringTok{"DOSE"}\NormalTok{, }\StringTok{"clusterProfiler"}\NormalTok{, }\StringTok{\textquotesingle{}DESeq2\textquotesingle{}}\NormalTok{, }\StringTok{\textquotesingle{}edgeR\textquotesingle{}}\NormalTok{, }
               \StringTok{\textquotesingle{}limma\textquotesingle{}}\NormalTok{, }\StringTok{"topGO"}\NormalTok{, }\StringTok{\textquotesingle{}Rgraphviz\textquotesingle{}}\NormalTok{, }\StringTok{\textquotesingle{}org.Hs.eg.db\textquotesingle{}}\NormalTok{)}
\ControlFlowTok{for}\NormalTok{ (p }\ControlFlowTok{in}\NormalTok{ bioc\_pack) \{ }
  \FunctionTok{cat}\NormalTok{(p, }\StringTok{\textquotesingle{}=========}\SpecialCharTok{\textbackslash{}n}\StringTok{\textquotesingle{}}\NormalTok{) }
  \ControlFlowTok{if}\NormalTok{ (}\SpecialCharTok{!}\FunctionTok{requireNamespace}\NormalTok{(p, }\AttributeTok{quietly =}\NormalTok{ T)) }
\NormalTok{    BiocManager}\SpecialCharTok{::}\FunctionTok{install}\NormalTok{(p, }\AttributeTok{update =}\NormalTok{ F, }\AttributeTok{ask =}\NormalTok{F) }
\NormalTok{\}}
\end{Highlighting}
\end{Shaded}

\begin{enumerate}
\def\labelenumi{\arabic{enumi}.}
\setcounter{enumi}{1}
\tightlist
\item
  安装LZ包(此包会不定时更新,后续更新只需要重新运行这句即可,上面的包不需要重新安装)
\end{enumerate}

\begin{Shaded}
\begin{Highlighting}[]
\CommentTok{\# 安装LZ包}
\NormalTok{devtools}\SpecialCharTok{::}\FunctionTok{install\_github}\NormalTok{(}\StringTok{"ArronLZ/LZ"}\NormalTok{, }\AttributeTok{upgrade =}\StringTok{"never"}\NormalTok{, }\AttributeTok{force =}\NormalTok{ T, }
                         \AttributeTok{build\_vignettes =}\NormalTok{ T)}
\CommentTok{\# 查看文档(点击LZ Documents查看网页版)}
\FunctionTok{vignette}\NormalTok{(}\StringTok{\textquotesingle{}LZ\textquotesingle{}}\NormalTok{)}
\end{Highlighting}
\end{Shaded}

\chapter{RNAseq差异分析}\label{deg}

\section{差异分析}\label{deg-mian}

\subsection{加载包}\label{ux52a0ux8f7dux5305}

\begin{Shaded}
\begin{Highlighting}[]
\FunctionTok{rm}\NormalTok{(}\AttributeTok{list =} \FunctionTok{ls}\NormalTok{());}\FunctionTok{gc}\NormalTok{()}
\FunctionTok{library}\NormalTok{(LZ)}
\FunctionTok{library}\NormalTok{(tibble)}
\FunctionTok{library}\NormalTok{(data.table)}
\FunctionTok{library}\NormalTok{(DESeq2)}
\FunctionTok{library}\NormalTok{(parallel)}
\FunctionTok{library}\NormalTok{(BiocParallel)}
\FunctionTok{library}\NormalTok{(ggplot2)}
\FunctionTok{cat}\NormalTok{(}\StringTok{" 您电脑线程为:"}\NormalTok{, }\FunctionTok{detectCores}\NormalTok{())}
\CommentTok{\# 如果是12代以后的interCPU,建议最高不超过6或8。服务器可加大设置,}
\CommentTok{\#  但不能大于线程总数。}
\CommentTok{\# 此处如果电脑性能一般,建议直接使用n=4或者2。}
\NormalTok{n }\OtherTok{=} \DecValTok{4}
\CommentTok{\# register(MulticoreParam(n)) 苹果和linux电脑使用这句替代下句}
\FunctionTok{register}\NormalTok{(}\FunctionTok{SnowParam}\NormalTok{(n))}
\end{Highlighting}
\end{Shaded}

\subsection{数据准备}\label{ux6570ux636eux51c6ux5907}

RNAseq下游分析必须准备两个文件:表达矩阵表格文件,分组表格文件
将gene\_count.csv,group.csv放在工作目录下

\begin{itemize}
\tightlist
\item
  gene\_count.csv 矩阵数据格式(数值型,整型)
\end{itemize}

\begin{longtable}[]{@{}lllll@{}}
\toprule\noalign{}
ID & sample1 & sample2 & sample3 & sample4 \\
\midrule\noalign{}
\endhead
\bottomrule\noalign{}
\endlastfoot
gene1 & 34 & 23 & 56 & 23 \\
gene2 & 35 & 23 & 12 & 23 \\
gene3 & 12 & 78 & 78 & 78 \\
\end{longtable}

\begin{itemize}
\tightlist
\item
  group.csv 分组数据格式:需要组的行名 包含于 表达谱的列名 rownames(group) \%in\% colname(eset)
\end{itemize}

\begin{longtable}[]{@{}lll@{}}
\toprule\noalign{}
Sample & Type & BATCH \\
\midrule\noalign{}
\endhead
\bottomrule\noalign{}
\endlastfoot
sample1 & tumor & 1 \\
sample2 & tumor & 1 \\
sample3 & normal & 2 \\
sample4 & normal & 2 \\
\end{longtable}

\begin{itemize}
\tightlist
\item
  注意:
  表达文件中的基因名是SYMBOL还是ESembleID。如为EsembleID,要注意是有小数点的ID还是没有小数点的。
  有小数点的形式为这样:ESEM00000123.34,没有点的是ESEM00000123。
  还有记住基因名这列的列名,建议统一设置为ID。
\end{itemize}

\subsection{文件夹准备}\label{ux6587ux4ef6ux5939ux51c6ux5907}

本文中有时也将文件夹称为目录,这两者等价。建议每个项目新建一个文件夹,例如本项目新建了一个名为
LZexample的文件夹,然后再在这个文件夹下建了一个data文件夹,以后data目录专门用来存原始文件,例
如RNAseq分析所需的eset.csv, group.csv或者更加原始的文件。

\textbf{目录结构建议}:本项目的目录初始结构,建议每个项目按着这个形式来。项目文件夹LZexample这个文
件夹名建议改成有意义的名称,一眼便能看出这个项目是什么数据或者什么目的,而data文件夹名不建议更改。
图片

\subsection{差异分析预设置}\label{ux5deeux5f02ux5206ux6790ux9884ux8bbeux7f6e}

\begin{Shaded}
\begin{Highlighting}[]
\CommentTok{\# 设置工作目录,即之后所有的操作如果不指定文件夹,都将会在这个文件夹下进行}
\FunctionTok{setwd}\NormalTok{(}\StringTok{"C:/data/LZexample"}\NormalTok{)  }\CommentTok{\# 按需更改成你的项目文件夹}
\FunctionTok{getwd}\NormalTok{() }\CommentTok{\# 检查是否更改工作目录成功了?}
\CommentTok{\# Windows系统下默认的文件夹路径是 "C:/data/LZexample" 这中斜杆分隔文件夹,}
\CommentTok{\# 如果是直接从win复制而来的,请将斜杠\textbackslash{}改成反斜杠/, 就如下面设置的这样。(改成\textbackslash{}\textbackslash{}也行)}

\CommentTok{\# 设置此次分析的标记}
\NormalTok{mark }\OtherTok{\textless{}{-}} \StringTok{"T\_C"}  \CommentTok{\# 此次分析的标记(记录谁比谁或和筛选阈值,建议设置的有意义)}
\CommentTok{\# 设置结果输出的文件夹,按需设定,可保持默认。}
\CommentTok{\#  第一次分析可以不用改,但如是第二分析,必须至少修改mark,outdir其中一个,}
\CommentTok{\#  否则会覆盖第一次的结果。}
\NormalTok{outdir }\OtherTok{\textless{}{-}} \StringTok{"result"}  
\NormalTok{outdirsub }\OtherTok{\textless{}{-}} \FunctionTok{paste0}\NormalTok{(outdir, }\StringTok{"/"}\NormalTok{,mark)}
\NormalTok{outdirsub.gsea }\OtherTok{\textless{}{-}} \FunctionTok{paste0}\NormalTok{(outdirsub, }\StringTok{"/gsea"}\NormalTok{)}
\NormalTok{outdirsub.rich }\OtherTok{\textless{}{-}} \FunctionTok{paste0}\NormalTok{(outdirsub, }\StringTok{"/rich"}\NormalTok{)}
\CommentTok{\# 按以上设置,结果将会保存在当前工作目录下的result/T\_C文件夹下}

\CommentTok{\# 差异分析阈值设定}
\NormalTok{ffdr }\OtherTok{\textless{}{-}} \FloatTok{0.1}
\NormalTok{fpval }\OtherTok{\textless{}{-}} \FloatTok{0.05}
\NormalTok{flogfc }\OtherTok{\textless{}{-}} \DecValTok{1}
\end{Highlighting}
\end{Shaded}

\subsection{差异分析}\label{ux5deeux5f02ux5206ux6790}

\begin{Shaded}
\begin{Highlighting}[]
\CommentTok{\# 读取并整理数据(如果都是按照上面的要求来的,不需要改这里的参数)}
\NormalTok{glist }\OtherTok{\textless{}{-}} \FunctionTok{DEG\_prepareData}\NormalTok{(}\AttributeTok{eset\_file =} \StringTok{"data/eset.csv"}\NormalTok{, }\CommentTok{\#表达数据的相对路径}
                         \AttributeTok{group\_file =} \StringTok{"data/group.csv"}\NormalTok{, }\CommentTok{\#分组文件的相对路径}
                         \AttributeTok{id\_dot =}\NormalTok{ F,  }\CommentTok{\# ESEM是否有点,有点设为T}
                         \AttributeTok{col.by =} \StringTok{"ID"}\NormalTok{,  }\CommentTok{\# 基因名列的列名}
                         \AttributeTok{annot\_trans =}\NormalTok{ F, }\CommentTok{\# 是否要注释,如果是EsembleID就需要设置为T}
                         \AttributeTok{f\_mark =}\NormalTok{ mark)}

\CommentTok{\# 差异分析 deseq2三部曲}
\NormalTok{dds }\OtherTok{\textless{}{-}} \FunctionTok{DEG\_DESeq2.dds}\NormalTok{(}\AttributeTok{exprset.group=}\NormalTok{glist, }\AttributeTok{batch =}\NormalTok{ F)}
\FunctionTok{DEG\_DESeq2.pca}\NormalTok{(dds, }\AttributeTok{outdir =}\NormalTok{ outdirsub) }\CommentTok{\# 此处有warning信息,不用管。}
\NormalTok{dds\_list }\OtherTok{\textless{}{-}} \FunctionTok{DEG\_DESeq2.ana}\NormalTok{(dds)}

\CommentTok{\# 差异后分析}
\CommentTok{\# 构建GSEA官网软件分析所需格式文件 \#此处会有warning,不用管}
\FunctionTok{DEGres\_ToGSEA}\NormalTok{(}\AttributeTok{diffan.obj =}\NormalTok{ dds\_list, }\AttributeTok{outdir =}\NormalTok{ outdirsub.gsea) }
\CommentTok{\# all\_father中记录了}
\CommentTok{\#             差异分析的总表,默认阈值的差异基因表,}
\CommentTok{\#             上调基因列表,下调基因列表}
\CommentTok{\#             以及R{-}GSEA分析所需要的所有mRNA的表达排序列表。}
\CommentTok{\#  是我们后续各种分析的万恶之源(因此命名all\_father)}
\CommentTok{\#  上述数据同时保存于当前工作目录/outdirsub.rich,}
\CommentTok{\#  文件是一个多sheet的xlsx表}
\NormalTok{all\_father }\OtherTok{\textless{}{-}} \FunctionTok{DEGres\_ToRICH}\NormalTok{(}\AttributeTok{diffan.obj =}\NormalTok{ dds\_list, }\AttributeTok{p=}\NormalTok{fpval, }\AttributeTok{q=}\NormalTok{ffdr, }
                            \AttributeTok{f=}\NormalTok{flogfc, }\AttributeTok{mark=}\NormalTok{mark, }\AttributeTok{outdir =}\NormalTok{ outdirsub.rich)}
\FunctionTok{save.image}\NormalTok{(}\AttributeTok{file =} \FunctionTok{paste0}\NormalTok{(outdirsub, }\StringTok{"/1.diff.img.RDATA"}\NormalTok{)) }\CommentTok{\# 保存中间数据}
\end{Highlighting}
\end{Shaded}

\section{火山图}\label{deg-valcano}

需要修改的是以下几个值:
df\_valcano:文件读取时的文件路径
ffdr: FDR阈值
fpval: PValue阈值
flogfc: logFC阈值
filterc: 火山图展示模式(p,fdr均考虑模式, 仅考虑p值模式,仅考虑fdr值模式,默认为第一种''fppadj'')
label\_gene: 展示基因列表
不清楚建议默认:fdr=0.1, pval=0.05, p,fdr均考虑模式。仅修改\texttt{label\_gene:\ 展示基因列表}即可。

\begin{Shaded}
\begin{Highlighting}[]
\CommentTok{\# 本流程中不需要运行,后续想再次分析时可从此步开始}
\CommentTok{\# load("./result/T\_C/1.diff.img.RDATA") }
\CommentTok{\# library(LZ)}
\FunctionTok{library}\NormalTok{(ggpubr);}\FunctionTok{library}\NormalTok{(ggrepel);}\FunctionTok{library}\NormalTok{(ggsci);}\FunctionTok{library}\NormalTok{(scales)}
\FunctionTok{library}\NormalTok{(tidyverse);}\FunctionTok{library}\NormalTok{(dplyr);}\FunctionTok{library}\NormalTok{(pheatmap);}\FunctionTok{library}\NormalTok{(RColorBrewer)}
\CommentTok{\# 导入火山图需要的数据,即差异分析后的未筛选表格(我们也称这个对象为resdf,}
\CommentTok{\#  resdf文件涵盖差异分析的所有结果信息,可以做后续所有基于差异分析或者基因}
\CommentTok{\#  列表的所有分析,如果后续分析时使用其它数据,请按这个resdf的格式改数据,}
\CommentTok{\#  主要就是把数据的列名改成和resdf的列名相同,即可用此包的函数分析画图)}
\CommentTok{\#  即df\_valcano \textless{}{-} readxl::read\_xlsx("xxx.xlsx", sheet = 1)}
\NormalTok{df\_valcano }\OtherTok{\textless{}{-}}\NormalTok{ all\_father}\SpecialCharTok{$}\NormalTok{DIFF.ALL}
\FunctionTok{names}\NormalTok{(df\_valcano) }\CommentTok{\# 对应的列名必须为Gene, log2FC, PValue, FDR}
\CommentTok{\# 差异分析阈值设定}
\NormalTok{ffdr }\OtherTok{\textless{}{-}} \FloatTok{0.1}
\NormalTok{fpval }\OtherTok{\textless{}{-}} \FloatTok{0.05}
\NormalTok{flogfc }\OtherTok{\textless{}{-}} \DecValTok{1}
\CommentTok{\# 模式设定}
\CommentTok{\# pvalue, padj均考虑模式 ("fpadj":仅考虑fdr值模式, "other": 仅考虑p值模式)}
\NormalTok{filterc }\OtherTok{\textless{}{-}} \StringTok{"fppadj"} 
\CommentTok{\# 设定需要标记的marker gene}
\NormalTok{label\_gene }\OtherTok{\textless{}{-}} \FunctionTok{c}\NormalTok{(}\StringTok{\textquotesingle{}TFRC\textquotesingle{}}\NormalTok{, }\StringTok{\textquotesingle{}ACSL1\textquotesingle{}}\NormalTok{, }\StringTok{\textquotesingle{}LPCAT3\textquotesingle{}}\NormalTok{, }\StringTok{\textquotesingle{}PCBP1\textquotesingle{}}\NormalTok{, }\StringTok{\textquotesingle{}FTH1\textquotesingle{}}\NormalTok{, }\StringTok{\textquotesingle{}SLC11A2\textquotesingle{}}\NormalTok{,}
                \StringTok{\textquotesingle{}SLC39A8\textquotesingle{}}\NormalTok{, }\StringTok{\textquotesingle{}SAT1\textquotesingle{}}\NormalTok{, }\StringTok{\textquotesingle{}FTL\textquotesingle{}}\NormalTok{, }\StringTok{\textquotesingle{}GSS\textquotesingle{}}\NormalTok{)}
\CommentTok{\# 查看想展示的基因在不在差异分析总表中}
\CommentTok{\# label\_gene \%in\% df\_valcano$Gene \%\textgreater{}\% all}
\CommentTok{\# pic\_data \%\textgreater{}\% filter(Row.names \%in\% label\_gene)}
\CommentTok{\# 火山图数据预处理}
\NormalTok{pic\_data }\OtherTok{\textless{}{-}} \FunctionTok{DEGp\_prepareVolcano}\NormalTok{(}\AttributeTok{df\_valcano =}\NormalTok{ df\_valcano, }\AttributeTok{filterc =}\NormalTok{ filterc)}
\CommentTok{\# 火山图 无标记}
\FunctionTok{DEGp\_Volcano}\NormalTok{(}\AttributeTok{result =}\NormalTok{ pic\_data, }\AttributeTok{logFC =}\NormalTok{ flogfc,}
             \AttributeTok{adj\_P =}\NormalTok{ ffdr, }\AttributeTok{label\_geneset =} \ConstantTok{NULL}\NormalTok{)}
\FunctionTok{ggsave}\NormalTok{(}\FunctionTok{paste0}\NormalTok{(outdirsub,}\StringTok{"/valcano.pdf"}\NormalTok{), }\AttributeTok{width =} \DecValTok{7}\NormalTok{, }\AttributeTok{height =} \DecValTok{7}\NormalTok{) }\CommentTok{\# 保存}
\CommentTok{\# 火山图 有标记}
\FunctionTok{DEGp\_Volcano}\NormalTok{(}\AttributeTok{result =}\NormalTok{ pic\_data, }\AttributeTok{logFC =}\NormalTok{ flogfc, }\CommentTok{\# log2(2)}
             \AttributeTok{adj\_P =}\NormalTok{ ffdr, }\AttributeTok{label\_geneset =}\NormalTok{ label\_gene) }\SpecialCharTok{\%\textgreater{}\%}
  \FunctionTok{ggplotGrob}\NormalTok{() }\SpecialCharTok{\%\textgreater{}\%}\NormalTok{ cowplot}\SpecialCharTok{::}\FunctionTok{plot\_grid}\NormalTok{()}
\FunctionTok{ggsave}\NormalTok{(}\FunctionTok{paste0}\NormalTok{(outdirsub, }\StringTok{"/valcano.mark.gene.pdf"}\NormalTok{), }\AttributeTok{width =} \DecValTok{7}\NormalTok{, }\AttributeTok{height =} \DecValTok{7}\NormalTok{) }
\end{Highlighting}
\end{Shaded}

\section{热图}\label{ux70edux56fe}

On the way \ldots{}

\chapter{富集分析}\label{enrich}

\section{GO \& KEGG 富集分析}\label{enrich-auto}

\subsection{一键脚本(批量处理)}\label{ux4e00ux952eux811aux672cux6279ux91cfux5904ux7406}

这是一个一键脚本,请新建一个单独的文件写这段脚本,然后按这个脚本的顶部注释
修改\texttt{resdf\ outd\ fc.list处即可},运行即可批量出不同FC的富集分析结果。

\begin{Shaded}
\begin{Highlighting}[]
\CommentTok{\# 此脚本为GO、KEGG分析(需要一个输入文件即可,为差异分析流程后的resdf文件)}
\CommentTok{\# 即为第一步(或1脚本)的结果的一个结果文件(DIFF\_an\_***.xlsx)}
\CommentTok{\# 即为resdf文件,此文件是差异分析后的总表}
\CommentTok{\# 注意如果采用了其他的分析方法得到差异分析后表,运行这个脚本时可能需要更改}
\CommentTok{\# 列名即我们的resdf对象的列名为Gene, log2FC,PValue,FDR,需要与这些个列名保}
\CommentTok{\# 持一致。}
\CommentTok{\# 此脚本中的需要修改的位于 /// *** /// 行中,另外还有一个LZ::setproxy()行,}
\CommentTok{\#   如果没有代理工具,或者代理工具不支持http代理,或者端口不通,请不要运行。}
\FunctionTok{rm}\NormalTok{(}\AttributeTok{list =} \FunctionTok{ls}\NormalTok{());}\FunctionTok{gc}\NormalTok{() }\CommentTok{\# 清空所有对象,慎用,必要时用}
\FunctionTok{suppressMessages}\NormalTok{(\{ }\FunctionTok{suppressWarnings}\NormalTok{(\{}
  \FunctionTok{library}\NormalTok{(LZ)}
  \FunctionTok{library}\NormalTok{(tidyverse);}\FunctionTok{library}\NormalTok{(data.table)}
  \FunctionTok{library}\NormalTok{(clusterProfiler);}\FunctionTok{library}\NormalTok{(enrichplot)}
  \FunctionTok{library}\NormalTok{(topGO);}\FunctionTok{library}\NormalTok{(Rgraphviz)}
  \FunctionTok{library}\NormalTok{(RColorBrewer);}\FunctionTok{library}\NormalTok{(ggsci);}\FunctionTok{library}\NormalTok{(pheatmap)}
  \FunctionTok{library}\NormalTok{(readxl)}
\NormalTok{\}) \})}
\CommentTok{\# 若无代理工具,切勿运行 }
\CommentTok{\# LZ::setproxy() \# 高危!!!新手不要运行此行,会使当前窗口断网!!!}
\CommentTok{\# Sys.getenv(\textquotesingle{}http\_proxy\textquotesingle{}) Sys.setenv(\textquotesingle{}http\_proxy\textquotesingle{}=\textquotesingle{}\textquotesingle{}) Sys.setenv(\textquotesingle{}https\_proxy\textquotesingle{}=\textquotesingle{}\textquotesingle{})}

\CommentTok{\# 如果是自己提供的表格,要按需修改列名为标准的resdf格式的列名:}
\CommentTok{\# 即: 表格必须含列名Gene, log2FC, PValue, FDR这四列,列名必须为这四个,}
\CommentTok{\# 提前在xlsx中修改好,然后取消下面这句的注释符,运行}
\CommentTok{\# resdf \textless{}{-} readxl::read\_xlsx("result/rnaseqOR{-}NC/rich/DIFF.an\_OR{-}NC.xlsx",}
\CommentTok{\#                  sheet = 1) \%\textgreater{}\% as.data.frame()}

\CommentTok{\# 按流程跑下来是运行这句,注意这句和上面的注释掉的是二选一,不要重复运行}
\NormalTok{resdf }\OtherTok{\textless{}{-}}\NormalTok{ all\_father}\SpecialCharTok{$}\NormalTok{DIFF.ALL}
\CommentTok{\# 输出目录}
\NormalTok{outd }\OtherTok{=} \StringTok{"result/xx/rich"} 
\CommentTok{\# logFC阈值, 多个阈值的话,}
\CommentTok{\#   写成fc.list \textless{}{-} list(\textquotesingle{}1.2\textquotesingle{} = log2(1.2), \textquotesingle{}2\textquotesingle{} = log2(2))}
\CommentTok{\# 注意!!!!!!:括号里log2(2)的2,和引号里\textquotesingle{}2\textquotesingle{}的2都要需同步要改。!!!}
\CommentTok{\# 否则可能会覆盖结果 }
\CommentTok{\# logFC阈值, 多个阈值}
\NormalTok{fc.list }\OtherTok{\textless{}{-}} \FunctionTok{list}\NormalTok{(}\StringTok{\textquotesingle{}1.5\textquotesingle{}} \OtherTok{=} \FunctionTok{log2}\NormalTok{(}\FloatTok{1.5}\NormalTok{), }\StringTok{\textquotesingle{}2\textquotesingle{}} \OtherTok{=} \FunctionTok{log2}\NormalTok{(}\DecValTok{2}\NormalTok{), }\StringTok{\textquotesingle{}4\textquotesingle{}} \OtherTok{=} \FunctionTok{log2}\NormalTok{(}\DecValTok{4}\NormalTok{))}
\CommentTok{\# logFC阈值, 单个阈值运行这句,也是二选一,不要重复运行}
\CommentTok{\# fc.list \textless{}{-} list(\textquotesingle{}2\textquotesingle{}=log2(2))}
\CommentTok{\# 设置物种为人类(如是人类则不需要更改)}
\NormalTok{GO\_database }\OtherTok{\textless{}{-}} \StringTok{\textquotesingle{}org.Hs.eg.db\textquotesingle{}} \CommentTok{\# keytypes(org.Hs.eg.db)}
\NormalTok{KEGG\_database }\OtherTok{\textless{}{-}} \StringTok{\textquotesingle{}hsa\textquotesingle{}} 

\CommentTok{\# 预处理数据符合GOKEGG分析的要求}
\CommentTok{\# \# 不同fc条件下的GOgenelist list(ALL, UP, DOWN)}
\NormalTok{gogenelist }\OtherTok{\textless{}{-}} \FunctionTok{lapply}\NormalTok{(fc.list, }\ControlFlowTok{function}\NormalTok{(x) \{}
  \FunctionTok{DEG\_prepareGOglist}\NormalTok{(resdf, }\AttributeTok{logfc =}\NormalTok{ x, }\AttributeTok{p =} \FloatTok{0.05}\NormalTok{, }\AttributeTok{fdr =} \FloatTok{0.1}\NormalTok{)}\ErrorTok{)}\NormalTok{ \}}
\CommentTok{\# gogenelist \%\textgreater{}\% length()}
\CommentTok{\# 对logFC迭代,每个FC新建一个目录,用来存upgene, downgene, allgene的GO结果}
\NormalTok{enrich }\OtherTok{\textless{}{-}} \FunctionTok{DEG\_runENRICH}\NormalTok{(}\AttributeTok{genelist =}\NormalTok{ gogenelist, }\AttributeTok{outdir =}\NormalTok{ outd, }
                        \AttributeTok{glist.save =}\NormalTok{ T, }\AttributeTok{rungo =}\NormalTok{ T, }\AttributeTok{runkegg =}\NormalTok{ T, }\AttributeTok{rapid =}\NormalTok{ T)}
\end{Highlighting}
\end{Shaded}

\subsection{简易GO,KEGG一次分析}\label{enrich-simple}

如果已经得到了差异基因列表,且无需批量分析,可以进行这个简易分析。
数据格式: \texttt{head(genelist.lh)}
{[}1{]} ``AARS1'' ``AATF'' ``ABCB7'' ``ABCE1'' ``ABHD11'' ``ABHD12''

\begin{Shaded}
\begin{Highlighting}[]
\CommentTok{\# 简易GO,KEGG一次分析(即:已经得到了差异基因列表)}
\CommentTok{\# LZ::setproxy() \# 代理设置,新手别碰,会断网}
\CommentTok{\# 差异基因列表}
\NormalTok{genelist.lh }\OtherTok{\textless{}{-}}\NormalTok{ pic.list}\SpecialCharTok{$}\NormalTok{sig.data}\SpecialCharTok{$}\NormalTok{Gene }
\CommentTok{\# 转换ID}
\NormalTok{gene\_df }\OtherTok{\textless{}{-}} \FunctionTok{bitr}\NormalTok{(genelist.lh, }\AttributeTok{fromType =} \StringTok{"SYMBOL"}\NormalTok{, }\AttributeTok{toType =} \FunctionTok{c}\NormalTok{(}\StringTok{"ENTREZID"}\NormalTok{, }\StringTok{"UNIPROT"}\NormalTok{), }
                \AttributeTok{OrgDb =} \StringTok{\textquotesingle{}org.Hs.eg.db\textquotesingle{}}\NormalTok{)}
\CommentTok{\# GO分析}
\NormalTok{go.lh }\OtherTok{\textless{}{-}} \FunctionTok{DEG\_GO}\NormalTok{(gene\_df, }\AttributeTok{orgdb =} \StringTok{"org.Hs.eg.db"}\NormalTok{, }\AttributeTok{sigNodes =} \DecValTok{20}\NormalTok{, }
                \AttributeTok{resultdir=}\StringTok{"./result/proteinOR{-}NC"}\NormalTok{, }\AttributeTok{filemark =} \StringTok{"p1.5\_g\_2"}\NormalTok{)}
\NormalTok{go.lhdf }\OtherTok{\textless{}{-}} \FunctionTok{sapply}\NormalTok{(go.lh, }\ControlFlowTok{function}\NormalTok{(x) x}\SpecialCharTok{@}\NormalTok{result, }\AttributeTok{simplify =}\NormalTok{ T)}
\FunctionTok{write\_xlsx}\NormalTok{(go.lhdf, }\AttributeTok{path =} \StringTok{"./result/xx/lh\_go.all.xlsx"}\NormalTok{)}
\CommentTok{\# KEGG分析}
\NormalTok{kegg.lh }\OtherTok{\textless{}{-}} \FunctionTok{DEG\_KEGG}\NormalTok{(gene\_df)}
\FunctionTok{write\_xlsx}\NormalTok{(kegg.lh}\SpecialCharTok{$}\NormalTok{pSigDF, }\AttributeTok{path =} \StringTok{"./result/xx/lh\_kegg.all.xlsx"}\NormalTok{)}
\end{Highlighting}
\end{Shaded}

\subsection{GO、KEGG分析结果可视化 \{\#enrich-visual\}}\label{dotplot}

\begin{Shaded}
\begin{Highlighting}[]
\CommentTok{\# dotplot go}
\CommentTok{\# 读取go分析保存的表格}
\CommentTok{\#dotData \textless{}{-} go$GODF$"倍数"$变化趋势(BP)}
\CommentTok{\# 自己提供表格读取}
\CommentTok{\#dotData \textless{}{-} readxl::read\_xlsx("kegg.xlsx", sheet = 1) }
\NormalTok{dotData }\OtherTok{\textless{}{-}}\NormalTok{ enrich}\SpecialCharTok{$}\NormalTok{GODF}\SpecialCharTok{$}\StringTok{"2"}\SpecialCharTok{$}\NormalTok{all}
\CommentTok{\# 筛选数据(按需配合其他筛选)}
\NormalTok{dotData }\OtherTok{\textless{}{-}} \FunctionTok{DEGp\_prepareDotplot}\NormalTok{(dotData, }\AttributeTok{head =} \DecValTok{30}\NormalTok{, }\AttributeTok{delete =} \ConstantTok{NULL}\NormalTok{)}
\NormalTok{pic.dot }\OtherTok{\textless{}{-}} \FunctionTok{DEGp\_Dotplot}\NormalTok{(dotData, }\AttributeTok{title =} \StringTok{\textquotesingle{}TOP of GO\textquotesingle{}}\NormalTok{, }
                        \AttributeTok{resultdir =} \StringTok{"./result/proteinOR{-}NC"}\NormalTok{, }
                        \AttributeTok{filemark =} \StringTok{\textquotesingle{}GO\_top\textquotesingle{}}\NormalTok{, }
                        \AttributeTok{pic.save =}\NormalTok{ T)}
\CommentTok{\# dotplot kegg}
\CommentTok{\# 读取kegg分析保存的表格,格式要求}
\CommentTok{\# 必须要有这四列Description, GeneRatio, pvalue, qvalue。Count列可有也可无。}
\CommentTok{\# 读取表格}
\CommentTok{\# dotDatak \textless{}{-} readxl::read\_xlsx("./result/proinOR/kegg.xlsx", sheet = 1)}

\NormalTok{dotDatak }\OtherTok{\textless{}{-}}\NormalTok{ enrich}\SpecialCharTok{$}\NormalTok{KEGGDF}\SpecialCharTok{$}\StringTok{\textquotesingle{}2\textquotesingle{}}\SpecialCharTok{$}\NormalTok{up}
\CommentTok{\# 筛选数据(按需配合其他筛选)}
\NormalTok{dotDataK }\OtherTok{\textless{}{-}} \FunctionTok{DEGp\_prepareDotplot}\NormalTok{(dotDatak, }\AttributeTok{head =} \DecValTok{30}\NormalTok{, }\AttributeTok{delete =} \ConstantTok{NULL}\NormalTok{)}
\NormalTok{pic.dotk }\OtherTok{\textless{}{-}} \FunctionTok{DEGp\_Dotplot}\NormalTok{(dotDataK, }\AttributeTok{title =} \StringTok{\textquotesingle{}TOP of KEGGpathway\textquotesingle{}}\NormalTok{, }
                         \AttributeTok{resultdir =} \StringTok{"./result/proteinOR{-}NC"}\NormalTok{, }
                         \AttributeTok{filemark =} \StringTok{\textquotesingle{}KEGG\_top\textquotesingle{}}\NormalTok{, }
                         \AttributeTok{pic.save =}\NormalTok{ F)}

\CommentTok{\# 组合图(可选运行,比例不是很好调整,单独出图AI内调整更自由)}
\NormalTok{gh }\OtherTok{\textless{}{-}} \FunctionTok{ggplotGrob}\NormalTok{(pic.dot)}
\NormalTok{gd }\OtherTok{\textless{}{-}} \FunctionTok{ggplotGrob}\NormalTok{(pic.dotk)}
\NormalTok{cowplot}\SpecialCharTok{::}\FunctionTok{plot\_grid}\NormalTok{(gh, gd, }\AttributeTok{rel\_widths =} \FunctionTok{c}\NormalTok{(}\DecValTok{1}\NormalTok{, }\FloatTok{1.25}\NormalTok{))}
\FunctionTok{ggsave}\NormalTok{(}\FunctionTok{paste0}\NormalTok{(dir\_out, }\StringTok{"/GO\_KEGG\_top.pdf"}\NormalTok{), }\AttributeTok{width =} \DecValTok{16}\NormalTok{, }\AttributeTok{height =} \DecValTok{10}\NormalTok{)}
\end{Highlighting}
\end{Shaded}

\section{GSEA分析}\label{enrich-gsea}

\subsection{R GSEA 批量分析}\label{r-gsea-ux6279ux91cfux5206ux6790}

\begin{itemize}
\item
  GSEA官网提供了GSEA分析软件和MSigDB数据库中的所有通路下载,如果需要更多的通路集可以自行下载,本包内置了MSigDB的H,C1-8的所有大类集及C2,C5的部分重要子类集,还有整理好的最新版的KEGG官方的PATHWAY通路集合。
\item
  本包构建了一个图形界面函数\texttt{runAPP\_GSEA()},可在安装完LZ包后直接通过运行\texttt{LZ::runAPP\_GSEA()}启动图形界面,也可在浏览器中打开。详细见\hyperref[enrich-gsea-r.gsea.gui]{R GSEA 图形界面}
\end{itemize}

\begin{Shaded}
\begin{Highlighting}[]
\FunctionTok{library}\NormalTok{(LZ)}
\FunctionTok{library}\NormalTok{(clusterProfiler)}
\FunctionTok{library}\NormalTok{(enrichplot)}
\FunctionTok{library}\NormalTok{(shiny)}
\FunctionTok{library}\NormalTok{(ggplot2)}

\CommentTok{\# 1. Gene list 排序表}
\NormalTok{genelist }\OtherTok{\textless{}{-}}\NormalTok{ all\_father}\SpecialCharTok{$}\NormalTok{gsealist}

\CommentTok{\# 2. Pathway Gene Set 表}
\CommentTok{\# 内置数据集 gmt.largelist.23.12.Hs.symbols}
\CommentTok{\#   含1. msigDB数据库的全部通路大类 }
\CommentTok{\#     2. msigDB的C2,C5的部分子集通路[这些子集是C2,C5的一部分] }
\CommentTok{\#     3.最新版本的KEGG全部通路}
\FunctionTok{data}\NormalTok{(}\StringTok{"gmt.largelist.23.12.Hs.symbols"}\NormalTok{)   }
\CommentTok{\# 选择KEGG通路集合,把美元符号后面字符删掉,然后按tab键可以选择其他数据集}
\NormalTok{gmt }\OtherTok{\textless{}{-}}\NormalTok{ gmt.largelist.}\DecValTok{23}\NormalTok{.}\FloatTok{12.}\NormalTok{Hs.symbols}\SpecialCharTok{$}\NormalTok{kegg.all.}\DecValTok{23}\NormalTok{.}\FloatTok{12.}\NormalTok{Hs.symbols.gmt}

\CommentTok{\# 全部该GeneSet数据的通路GSEA分析  {-}{-}{-}{-}{-}{-}{-}{-}{-}{-}{-}}
\NormalTok{gsea }\OtherTok{\textless{}{-}} \FunctionTok{DEG\_runGSEA}\NormalTok{(}\AttributeTok{genelist=}\NormalTok{genelist, }\AttributeTok{gmt\_set=}\NormalTok{gmt, }\AttributeTok{pic.save=}\NormalTok{F)}
\CommentTok{\# 将所有的分析结果导出到本地[gsea.result]文件夹,统计总结表名为gsea\_stat}
\CommentTok{\#   导出文件夹名和文件名均可按需修改}
\FunctionTok{DEGp\_GSEA\_plotALL}\NormalTok{(gsea, }\AttributeTok{result\_dir =} \StringTok{"gsea.result"}\NormalTok{, }
                  \AttributeTok{xl\_filename =} \StringTok{"gsea\_stat"}\NormalTok{)}

\CommentTok{\# 单个GeneSet数据的通路GSEA分析(且从自己准备的gmt文件开始),可从MSigDB网}
\CommentTok{\#  站搜索下载}
\CommentTok{\# 设定文件路径}
\NormalTok{gmt\_filename }\OtherTok{\textless{}{-}} \StringTok{"D:/Team/RNAseq/data/geneset/WP\_FERROPTOSIS.v2022.1.Hs.gmt"}
\NormalTok{gmt\_single }\OtherTok{\textless{}{-}}\NormalTok{ clusterProfiler}\SpecialCharTok{::}\FunctionTok{read.gmt}\NormalTok{(gmt\_filename)}
\NormalTok{gsea.single }\OtherTok{\textless{}{-}} \FunctionTok{DEG\_runGSEA}\NormalTok{(}\AttributeTok{genelist =}\NormalTok{ genelist, }\AttributeTok{gmt\_set =}\NormalTok{ gmt\_single, }
                           \AttributeTok{pic.save=}\NormalTok{T, }\AttributeTok{outdir =} \StringTok{"./gsea.result2/"}\NormalTok{, }
                           \AttributeTok{filename =} \StringTok{"ferr"}\NormalTok{)}
\end{Highlighting}
\end{Shaded}

\subsection{R GSEA 图形界面}\label{enrich-gsea-r.gsea.gui}

图形界面可以用自己的表格数据上传来做GSEA分析,表格必须有且仅有两列,分别
为Gene列和log2FC列,具体表格形式如下:

\begin{longtable}[]{@{}ll@{}}
\toprule\noalign{}
Gene & log2FC \\
\midrule\noalign{}
\endhead
\bottomrule\noalign{}
\endlastfoot
geneA & 9.8 \\
geneX & 4.3 \\
geneY & 1.2 \\
\ldots{} & \ldots{} \\
geneZ & 0.3 \\
geneB & -0.8 \\
geneZ & -5.2 \\
\end{longtable}

注意:虽然要求第二列名为log2FC,但第二列只要是表示变化倍数就可以,不一定要是log2后数据,没有排序也没关系。

\begin{Shaded}
\begin{Highlighting}[]
\CommentTok{\# 启动方法1(不带参数启动),适合自己已经有了差异分析结果的表格的情况,}
\CommentTok{\#  那么运行此句后,不需要在R里写任何代码,如果通过LZ::runAPP\_GSEA()甚至}
\CommentTok{\#  都不需要加载包。}
\CommentTok{\#  使用这种方法,则必须要上传上述指定的表格形式的表格后才能点击画图。}
\FunctionTok{runAPP\_GSEA}\NormalTok{() }\CommentTok{\# 或在安装成功LZ包后,直接通过LZ::runAPP\_GSEA()来启动}

\CommentTok{\# 启动方法2(带参数启动),适合在R中已经有genelist的情况。}
\CommentTok{\# 已在R里有了genelist的话,运行此句启动图形界面,这样就不需要上传表格}
\FunctionTok{runAPP\_GSEA}\NormalTok{(}\AttributeTok{genelist =}\NormalTok{ genelist)}
\end{Highlighting}
\end{Shaded}

\subsection{一些进阶操作及技巧(不会没有关系,不做过多解释,自行体会,无R语言基础者慎入)}\label{ux4e00ux4e9bux8fdbux9636ux64cdux4f5cux53caux6280ux5de7ux4e0dux4f1aux6ca1ux6709ux5173ux7cfbux4e0dux505aux8fc7ux591aux89e3ux91caux81eaux884cux4f53ux4f1aux65e0rux8bedux8a00ux57faux7840ux8005ux614eux5165}

\begin{Shaded}
\begin{Highlighting}[]
\CommentTok{\# 进阶操作 {-}{-}{-}{-}{-}{-}{-}{-}{-}{-}{-}{-}{-}{-}{-}{-}{-}{-}{-}{-}{-}{-}{-}{-}{-}{-}{-}}
\CommentTok{\# 1. 查找指定通路的图}
\CommentTok{\#  如果自己知道通路的名字,可以通过查找来定位到通路的位置,然后单独画图}
\NormalTok{pathway }\OtherTok{=} \StringTok{"\^{}ABC"}
\NormalTok{n }\OtherTok{=} \FunctionTok{grep}\NormalTok{(pathway, gsea[,}\StringTok{"ID"}\NormalTok{])}
\FunctionTok{grep}\NormalTok{(pathway, gsea[,}\StringTok{"ID"}\NormalTok{], }\AttributeTok{value =}\NormalTok{ T) }\CommentTok{\# 查看找到的通路名称,必须时唯一值,否则请使用更加详细的查找条件再查找一次。}
\FunctionTok{DEGp\_GSEA}\NormalTok{(gsea, }\AttributeTok{num =}\NormalTok{ n)}

\CommentTok{\# 2. 从当前的gmt文件中获取指定的Pathway Gene set(gmt对象名一定腰围gmt才行)}
\NormalTok{gmtdf.find }\OtherTok{\textless{}{-}} \FunctionTok{find\_pathway}\NormalTok{(}\StringTok{"\^{}Ferr"}\NormalTok{)}
\NormalTok{gsea.single }\OtherTok{\textless{}{-}} \FunctionTok{DEG\_runGSEA}\NormalTok{(}\AttributeTok{genelist =}\NormalTok{ genelist, }\AttributeTok{gmt\_set=}\NormalTok{gmtdf.find, }
                           \AttributeTok{pic.save=}\NormalTok{T, }\AttributeTok{outdir =} \StringTok{"./gsea.ytb3/"}\NormalTok{, }
                           \AttributeTok{filename =} \StringTok{"fer\_taget"}\NormalTok{)}

\CommentTok{\# 3. 如果自己的分析中想将自己的数据转化为GSEA要求的genelist数据,}
\CommentTok{\#    例如自己的分析项目中有一个名为re的对象,该对象中有}
\CommentTok{\#    基因名列XX 和 基因变化倍数列 YY}
\NormalTok{genelist }\OtherTok{\textless{}{-}}\NormalTok{ re}\SpecialCharTok{$}\NormalTok{YY}
\FunctionTok{names}\NormalTok{(genelist) }\OtherTok{\textless{}{-}}\NormalTok{ re}\SpecialCharTok{$}\NormalTok{XX}
\NormalTok{genelist }\OtherTok{\textless{}{-}} \FunctionTok{na.omit}\NormalTok{(}\FunctionTok{sort}\NormalTok{(genelist, }\AttributeTok{decreasing =}\NormalTok{ T))}

\CommentTok{\# 4. GSEA图形界面中若使用结束按钮关闭程序,会将最后一次画图的数据保留在}
\CommentTok{\#    R会话中,具体如下:}
\CommentTok{\# 最后画的一幅图}
\NormalTok{pic\_gsea}
\CommentTok{\# 保存图片}
\FunctionTok{pdf}\NormalTok{(}\StringTok{\textquotesingle{}aaa.pdf\textquotesingle{}}\NormalTok{, }\AttributeTok{width =} \DecValTok{6}\NormalTok{, }\AttributeTok{height =} \DecValTok{5}\NormalTok{)}
\NormalTok{pic\_gsea}
\FunctionTok{dev.off}\NormalTok{()}
\CommentTok{\# 最后一次选择的通路集详细}
\NormalTok{sy\_gmt\_taget}
\CommentTok{\# 最后一次选择的通路名称}
\NormalTok{sy\_pathway\_name}
\CommentTok{\# 最后一次选择的大类通路集详细的前六行}
\NormalTok{sy\_gmt }\SpecialCharTok{\%\textgreater{}\%} \FunctionTok{head}\NormalTok{()}

\CommentTok{\# 5. 转换成宽型文件}
\NormalTok{gmt.w2 }\OtherTok{\textless{}{-}} \FunctionTok{gmt\_longTowide2}\NormalTok{(gmt) }
\end{Highlighting}
\end{Shaded}

\subsection{GSEA官方软件}\label{gseaux5b98ux65b9ux8f6fux4ef6}

自行搜素方法,网络上有大量图文教程 \ldots{}

\chapter{差异及富集分析可视化专题}\label{visual}

On the way \ldots{}

\chapter{RNAseq上游流程}\label{rnaseq-rsubread}

On the way \ldots{}

\chapter{多组学}\label{multi-omics}

On the way

\chapter{CUT\&TAG}\label{cuttag}

On the way \ldots{}

\chapter{单细胞分析}\label{scRNA}

On the way \ldots{}

\chapter{空间转录组}\label{spatial}

On the way \ldots{}

\chapter*{References}\label{references}
\addcontentsline{toc}{chapter}{References}

  \bibliography{book.bib,packages.bib}

\end{document}
