% Options for packages loaded elsewhere
\PassOptionsToPackage{unicode}{hyperref}
\PassOptionsToPackage{hyphens}{url}
%
\documentclass[
]{book}
\usepackage{amsmath,amssymb}
\usepackage{iftex}
\ifPDFTeX
  \usepackage[T1]{fontenc}
  \usepackage[utf8]{inputenc}
  \usepackage{textcomp} % provide euro and other symbols
\else % if luatex or xetex
  \usepackage{unicode-math} % this also loads fontspec
  \defaultfontfeatures{Scale=MatchLowercase}
  \defaultfontfeatures[\rmfamily]{Ligatures=TeX,Scale=1}
\fi
\usepackage{lmodern}
\ifPDFTeX\else
  % xetex/luatex font selection
\fi
% Use upquote if available, for straight quotes in verbatim environments
\IfFileExists{upquote.sty}{\usepackage{upquote}}{}
\IfFileExists{microtype.sty}{% use microtype if available
  \usepackage[]{microtype}
  \UseMicrotypeSet[protrusion]{basicmath} % disable protrusion for tt fonts
}{}
\makeatletter
\@ifundefined{KOMAClassName}{% if non-KOMA class
  \IfFileExists{parskip.sty}{%
    \usepackage{parskip}
  }{% else
    \setlength{\parindent}{0pt}
    \setlength{\parskip}{6pt plus 2pt minus 1pt}}
}{% if KOMA class
  \KOMAoptions{parskip=half}}
\makeatother
\usepackage{xcolor}
\usepackage{color}
\usepackage{fancyvrb}
\newcommand{\VerbBar}{|}
\newcommand{\VERB}{\Verb[commandchars=\\\{\}]}
\DefineVerbatimEnvironment{Highlighting}{Verbatim}{commandchars=\\\{\}}
% Add ',fontsize=\small' for more characters per line
\usepackage{framed}
\definecolor{shadecolor}{RGB}{248,248,248}
\newenvironment{Shaded}{\begin{snugshade}}{\end{snugshade}}
\newcommand{\AlertTok}[1]{\textcolor[rgb]{0.94,0.16,0.16}{#1}}
\newcommand{\AnnotationTok}[1]{\textcolor[rgb]{0.56,0.35,0.01}{\textbf{\textit{#1}}}}
\newcommand{\AttributeTok}[1]{\textcolor[rgb]{0.13,0.29,0.53}{#1}}
\newcommand{\BaseNTok}[1]{\textcolor[rgb]{0.00,0.00,0.81}{#1}}
\newcommand{\BuiltInTok}[1]{#1}
\newcommand{\CharTok}[1]{\textcolor[rgb]{0.31,0.60,0.02}{#1}}
\newcommand{\CommentTok}[1]{\textcolor[rgb]{0.56,0.35,0.01}{\textit{#1}}}
\newcommand{\CommentVarTok}[1]{\textcolor[rgb]{0.56,0.35,0.01}{\textbf{\textit{#1}}}}
\newcommand{\ConstantTok}[1]{\textcolor[rgb]{0.56,0.35,0.01}{#1}}
\newcommand{\ControlFlowTok}[1]{\textcolor[rgb]{0.13,0.29,0.53}{\textbf{#1}}}
\newcommand{\DataTypeTok}[1]{\textcolor[rgb]{0.13,0.29,0.53}{#1}}
\newcommand{\DecValTok}[1]{\textcolor[rgb]{0.00,0.00,0.81}{#1}}
\newcommand{\DocumentationTok}[1]{\textcolor[rgb]{0.56,0.35,0.01}{\textbf{\textit{#1}}}}
\newcommand{\ErrorTok}[1]{\textcolor[rgb]{0.64,0.00,0.00}{\textbf{#1}}}
\newcommand{\ExtensionTok}[1]{#1}
\newcommand{\FloatTok}[1]{\textcolor[rgb]{0.00,0.00,0.81}{#1}}
\newcommand{\FunctionTok}[1]{\textcolor[rgb]{0.13,0.29,0.53}{\textbf{#1}}}
\newcommand{\ImportTok}[1]{#1}
\newcommand{\InformationTok}[1]{\textcolor[rgb]{0.56,0.35,0.01}{\textbf{\textit{#1}}}}
\newcommand{\KeywordTok}[1]{\textcolor[rgb]{0.13,0.29,0.53}{\textbf{#1}}}
\newcommand{\NormalTok}[1]{#1}
\newcommand{\OperatorTok}[1]{\textcolor[rgb]{0.81,0.36,0.00}{\textbf{#1}}}
\newcommand{\OtherTok}[1]{\textcolor[rgb]{0.56,0.35,0.01}{#1}}
\newcommand{\PreprocessorTok}[1]{\textcolor[rgb]{0.56,0.35,0.01}{\textit{#1}}}
\newcommand{\RegionMarkerTok}[1]{#1}
\newcommand{\SpecialCharTok}[1]{\textcolor[rgb]{0.81,0.36,0.00}{\textbf{#1}}}
\newcommand{\SpecialStringTok}[1]{\textcolor[rgb]{0.31,0.60,0.02}{#1}}
\newcommand{\StringTok}[1]{\textcolor[rgb]{0.31,0.60,0.02}{#1}}
\newcommand{\VariableTok}[1]{\textcolor[rgb]{0.00,0.00,0.00}{#1}}
\newcommand{\VerbatimStringTok}[1]{\textcolor[rgb]{0.31,0.60,0.02}{#1}}
\newcommand{\WarningTok}[1]{\textcolor[rgb]{0.56,0.35,0.01}{\textbf{\textit{#1}}}}
\usepackage{longtable,booktabs,array}
\usepackage{calc} % for calculating minipage widths
% Correct order of tables after \paragraph or \subparagraph
\usepackage{etoolbox}
\makeatletter
\patchcmd\longtable{\par}{\if@noskipsec\mbox{}\fi\par}{}{}
\makeatother
% Allow footnotes in longtable head/foot
\IfFileExists{footnotehyper.sty}{\usepackage{footnotehyper}}{\usepackage{footnote}}
\makesavenoteenv{longtable}
\usepackage{graphicx}
\makeatletter
\def\maxwidth{\ifdim\Gin@nat@width>\linewidth\linewidth\else\Gin@nat@width\fi}
\def\maxheight{\ifdim\Gin@nat@height>\textheight\textheight\else\Gin@nat@height\fi}
\makeatother
% Scale images if necessary, so that they will not overflow the page
% margins by default, and it is still possible to overwrite the defaults
% using explicit options in \includegraphics[width, height, ...]{}
\setkeys{Gin}{width=\maxwidth,height=\maxheight,keepaspectratio}
% Set default figure placement to htbp
\makeatletter
\def\fps@figure{htbp}
\makeatother
\setlength{\emergencystretch}{3em} % prevent overfull lines
\providecommand{\tightlist}{%
  \setlength{\itemsep}{0pt}\setlength{\parskip}{0pt}}
\setcounter{secnumdepth}{5}
\usepackage{ctex}

\usepackage{booktabs}
\usepackage{amsthm}
\makeatletter
\def\thm@space@setup{%
  \thm@preskip=8pt plus 2pt minus 4pt
  \thm@postskip=\thm@preskip
}
\usepackage{geometry}
\geometry{left=2.5cm,right=2.5cm,top=3cm,bottom=3cm}
\makeatother
\ifLuaTeX
  \usepackage{selnolig}  % disable illegal ligatures
\fi
\usepackage[]{natbib}
\bibliographystyle{apalike}
\IfFileExists{bookmark.sty}{\usepackage{bookmark}}{\usepackage{hyperref}}
\IfFileExists{xurl.sty}{\usepackage{xurl}}{} % add URL line breaks if available
\urlstyle{same}
\hypersetup{
  pdftitle={LZ 文档},
  pdfauthor={作者:荔枝 earth\_farmer@outlook.com},
  hidelinks,
  pdfcreator={LaTeX via pandoc}}

\title{LZ 文档}
\author{作者:荔枝 \href{mailto:earth_farmer@outlook.com}{\nolinkurl{earth\_farmer@outlook.com}}}
\date{更新:2023-11-17}

\begin{document}
\maketitle

{
\setcounter{tocdepth}{1}
\tableofcontents
}
\hypertarget{ux524dux8a00}{%
\chapter*{前言}\label{ux524dux8a00}}
\addcontentsline{toc}{chapter}{前言}

本包致力于简化生信分析流程和批量分析。目前主要为RNAseq分析流程,后期会加入多组学联合
分析流程。

\hypertarget{ux4e3aux4f55ux9605ux8bfbux672cux4e66}{%
\section*{为何阅读本书}\label{ux4e3aux4f55ux9605ux8bfbux672cux4e66}}
\addcontentsline{toc}{section}{为何阅读本书}

本书是为LZ R package写的使用文档。旨在让完全没有编程基础或R基础的人学会使用LZ包来进
行一些生信分析。LZ包致力于简化生信分析流程和批量分析。
目前LZ包已经完成了RNAseq下游分析流程的大部分,后期完善后还有加入更多的功能。例如
RNAseq的上有分析流程,多组学联合分析流程,单细胞分析流程。
通过学习完本书,您将会在不需要系统学习R语言的情况下快速分析测序数据,如有疑问请发
email至\texttt{earth\_farmer@outlook.com},尽量有答复但不予保证。

\hypertarget{ux672cux4e66ux7ed3ux6784}{%
\section*{本书结构}\label{ux672cux4e66ux7ed3ux6784}}
\addcontentsline{toc}{section}{本书结构}

\hypertarget{ux81f4ux8c22}{%
\section*{致谢}\label{ux81f4ux8c22}}
\addcontentsline{toc}{section}{致谢}

\hypertarget{install}{%
\chapter{安装}\label{install}}

\textbf{LZ} R包可以从Github上安装。
先安装R及Rstudio, 前者是核心,后者是编辑器(写代码的地方)。
1. 安装\href{https://www.r-project.org/}{R最新版}
2. 安装\href{https://posit.co/download/rstudio-desktop/}{Rstudio最新版}
3. Win电脑可以考虑安装R版本对应的\href{https://cran.r-project.org/bin/windows/Rtools/}{Rtools} (可选项)

\begin{Shaded}
\begin{Highlighting}[]
\CommentTok{\# 安装biocondutor}
\ControlFlowTok{if}\NormalTok{ (}\SpecialCharTok{!}\FunctionTok{require}\NormalTok{(}\StringTok{"BiocManager"}\NormalTok{, }\AttributeTok{quietly =} \ConstantTok{TRUE}\NormalTok{)) \{}
  \FunctionTok{install.packages}\NormalTok{(}\StringTok{"BiocManager"}\NormalTok{)}
  \CommentTok{\# R 4.2.0}
\NormalTok{  BiocManager}\SpecialCharTok{::}\FunctionTok{install}\NormalTok{(}\AttributeTok{version =} \StringTok{"3.18"}\NormalTok{) }\CommentTok{\# 4.3 == 3.18}
\NormalTok{\}}
\CommentTok{\# 设置镜像(可选)}
\FunctionTok{options}\NormalTok{(}\StringTok{"repos"} \OtherTok{=} \FunctionTok{c}\NormalTok{(}\AttributeTok{CRAN =} \StringTok{"https://mirrors.tuna.tsinghua.edu.cn/CRAN/"}\NormalTok{)) }
\FunctionTok{options}\NormalTok{(}\AttributeTok{BioC\_mirror=}\StringTok{"https://mirrors.tuna.tsinghua.edu.cn/bioconductor"}\NormalTok{)}
\CommentTok{\# 安装LZ依赖包}
\CommentTok{\# 安装cran包}
\NormalTok{cran\_pack }\OtherTok{\textless{}{-}} \FunctionTok{c}\NormalTok{(}\StringTok{\textquotesingle{}devtools\textquotesingle{}}\NormalTok{, }\StringTok{\textquotesingle{}prettydoc\textquotesingle{}}\NormalTok{, }\StringTok{\textquotesingle{}Hmisc\textquotesingle{}}\NormalTok{)}
\ControlFlowTok{for}\NormalTok{ (p }\ControlFlowTok{in}\NormalTok{ cran\_pack) \{ }\ControlFlowTok{if}\NormalTok{ (}\SpecialCharTok{!}\FunctionTok{requireNamespace}\NormalTok{(p, }\AttributeTok{quietly =}\NormalTok{ T)) }\FunctionTok{install.packages}\NormalTok{(p) \}}
\CommentTok{\# 安装bioconductor包}
\NormalTok{bioc\_pack }\OtherTok{\textless{}{-}} \FunctionTok{c}\NormalTok{(}\StringTok{"DOSE"}\NormalTok{, }\StringTok{"clusterProfiler"}\NormalTok{, }\StringTok{\textquotesingle{}DESeq2\textquotesingle{}}\NormalTok{, }\StringTok{\textquotesingle{}edgeR\textquotesingle{}}\NormalTok{, }\StringTok{\textquotesingle{}limma\textquotesingle{}}\NormalTok{)}
\ControlFlowTok{for}\NormalTok{ (p }\ControlFlowTok{in}\NormalTok{ bioc\_pack) \{ }
  \FunctionTok{cat}\NormalTok{(p, }\StringTok{\textquotesingle{}=========}\SpecialCharTok{\textbackslash{}n}\StringTok{\textquotesingle{}}\NormalTok{) }
  \ControlFlowTok{if}\NormalTok{ (}\SpecialCharTok{!}\FunctionTok{requireNamespace}\NormalTok{(p, }\AttributeTok{quietly =}\NormalTok{ T)) BiocManager}\SpecialCharTok{::}\FunctionTok{install}\NormalTok{(p, }\AttributeTok{update =}\NormalTok{ F, }\AttributeTok{ask =}\NormalTok{F) }
\NormalTok{  \}}
\CommentTok{\# 安装LZ包}
\NormalTok{devtools}\SpecialCharTok{::}\FunctionTok{install\_github}\NormalTok{(}\StringTok{"ArronLZ/LZ"}\NormalTok{, }\AttributeTok{upgrade =}\StringTok{"never"}\NormalTok{, }\AttributeTok{force =}\NormalTok{ T, }
                         \AttributeTok{build\_vignettes =}\NormalTok{ T)}
\end{Highlighting}
\end{Shaded}

\hypertarget{deg}{%
\chapter{RNAseq差异分析}\label{deg}}

\hypertarget{deg-mian}{%
\section{差异分析}\label{deg-mian}}

\hypertarget{ux52a0ux8f7dux5305}{%
\subsection{加载包}\label{ux52a0ux8f7dux5305}}

\begin{Shaded}
\begin{Highlighting}[]
\FunctionTok{rm}\NormalTok{(}\AttributeTok{list =} \FunctionTok{ls}\NormalTok{());}\FunctionTok{gc}\NormalTok{()}
\FunctionTok{library}\NormalTok{(LZ)}
\FunctionTok{library}\NormalTok{(tibble)}
\FunctionTok{library}\NormalTok{(parallel)}
\FunctionTok{library}\NormalTok{(data.table)}
\FunctionTok{library}\NormalTok{(DESeq2)}
\FunctionTok{cat}\NormalTok{(}\StringTok{" 您电脑线程为:"}\NormalTok{, }\FunctionTok{detectCores}\NormalTok{())}
\CommentTok{\# 如果是12代以后的interCPU,建议最高不超过6或8。服务器可加大设置,但不能大于线程总数。}
\CommentTok{\# 此处如果电脑一般,建议直接使用n=4或者2。}
\NormalTok{n }\OtherTok{=} \DecValTok{6}  
\CommentTok{\# register(MulticoreParam(n)) 苹果和linux电脑使用这句替代下句}
\FunctionTok{register}\NormalTok{(}\FunctionTok{SnowParam}\NormalTok{(n))}
\end{Highlighting}
\end{Shaded}

\hypertarget{ux8bbeux7f6eux8f93ux51faux6587ux4ef6ux5939}{%
\subsection{设置输出文件夹}\label{ux8bbeux7f6eux8f93ux51faux6587ux4ef6ux5939}}

结果数据均在result文件夹下,如果不懂,不要修改。但是如果多次运行的话,第二次及以后请
务必修改outdir,例如可改为\texttt{outdir\ \textless{}-\ "result2"},其余后面不需要修改。

\begin{Shaded}
\begin{Highlighting}[]
\NormalTok{mark }\OtherTok{\textless{}{-}} \StringTok{"OR{-}NC"}  \CommentTok{\# 此次差异分析的标记(记录谁比谁或和筛选阈值)}
\NormalTok{outdir }\OtherTok{\textless{}{-}} \StringTok{"result/rnaseq"}  \CommentTok{\# 按需设定(可保持默认,如果修改只修改此处即可,下面无需修改)}
\NormalTok{outdirsub }\OtherTok{\textless{}{-}} \FunctionTok{paste0}\NormalTok{(outdir, mark)}
\NormalTok{outdirsub.gsea }\OtherTok{\textless{}{-}} \FunctionTok{paste0}\NormalTok{(outdirsub, }\StringTok{"/gsea"}\NormalTok{)}
\NormalTok{outdirsub.rich }\OtherTok{\textless{}{-}} \FunctionTok{paste0}\NormalTok{(outdirsub, }\StringTok{"/rich"}\NormalTok{)}
\end{Highlighting}
\end{Shaded}

\hypertarget{ux5deeux5f02ux5206ux6790}{%
\subsection{差异分析}\label{ux5deeux5f02ux5206ux6790}}

将gene\_count.csv,group.csv放在工作目录下

\begin{itemize}
\tightlist
\item
  gene\_count.csv 矩阵数据格式(数值型,整型)
\end{itemize}

\begin{longtable}[]{@{}lllll@{}}
\toprule\noalign{}
gene & row1 & row2 & row3 & row4 \\
\midrule\noalign{}
\endhead
\bottomrule\noalign{}
\endlastfoot
gene1 & 34 & 23 & 56 & 23 \\
gene2 & 35 & 23 & 12 & 23 \\
gene3 & 12 & 78 & 78 & 78 \\
\end{longtable}

\begin{itemize}
\tightlist
\item
  group.csv 分组数据格式:需要组的行名 包含于 表达谱的列名 rownames(group) \%in\% colname(eset)
\end{itemize}

\begin{longtable}[]{@{}lll@{}}
\toprule\noalign{}
Sample & Type & BATCH \\
\midrule\noalign{}
\endhead
\bottomrule\noalign{}
\endlastfoot
rowname1 & tumor & 1 \\
rowname2 & tumor & 1 \\
rowname3 & normal & 2 \\
rowname4 & normal & 2 \\
\end{longtable}

\begin{Shaded}
\begin{Highlighting}[]
\NormalTok{glist }\OtherTok{\textless{}{-}} \FunctionTok{DEG\_prepareData}\NormalTok{(}\AttributeTok{eset\_file =} \StringTok{"gene\_count.csv"}\NormalTok{,}
                         \AttributeTok{group\_file =} \StringTok{"group.csv"}\NormalTok{,}
                         \AttributeTok{annot\_trans =}\NormalTok{ T,}
                         \AttributeTok{f\_mark =}\NormalTok{ mark)}
\CommentTok{\# 差异分析 deseq2三部曲}
\NormalTok{dds }\OtherTok{\textless{}{-}} \FunctionTok{DEG\_DESeq2.dds}\NormalTok{(}\AttributeTok{exprset.group=}\NormalTok{glist, }\AttributeTok{batch =}\NormalTok{ F)}
\FunctionTok{DEG\_DESeq2.pca}\NormalTok{(dds, }\AttributeTok{outdir =}\NormalTok{ outdirsub) }\CommentTok{\# 此处有warning信息,不用管。}
\NormalTok{dds\_list }\OtherTok{\textless{}{-}} \FunctionTok{DEG\_DESeq2.ana}\NormalTok{(dds)}

\CommentTok{\# 差异后分析}
\ControlFlowTok{if}\NormalTok{ ( }\FunctionTok{dir.exists}\NormalTok{(outdirsub.gsea) ) \{}
  \CommentTok{\# 如果outdirsub.gsea文件夹存在,清空该文件夹下所有文件}
  \FunctionTok{file.remove}\NormalTok{(}\FunctionTok{list.files}\NormalTok{(outdirsub.gsea, }\AttributeTok{full.names =}\NormalTok{ T))}
\NormalTok{\}}
\CommentTok{\# 构建GSEA官网软件分析所需格式文件}
\CommentTok{\# 此处有warning信息,不用管。}
\FunctionTok{DEGres\_ToGSEA}\NormalTok{(}\AttributeTok{diffan.obj =}\NormalTok{ dds\_list, }\AttributeTok{outdir =}\NormalTok{ outdirsub.gsea) }

\CommentTok{\# all\_father中记录了}
\CommentTok{\#             差异分析的总表,默认阈值的差异基因表,}
\CommentTok{\#             上调基因列表,下调基因列表}
\CommentTok{\#             以及R{-}GSEA分析所需要的所有mRNA的表达排序列表。}
\CommentTok{\#   是我们后续各种分析的万恶之源(因此命名all\_father)}
\CommentTok{\#   上述数据同时被保存在本地硬盘的到一个多sheet的xlsx表中【outdirsub.rich目录中】}
\NormalTok{all\_father }\OtherTok{\textless{}{-}} \FunctionTok{DEGres\_ToRICH}\NormalTok{(}\AttributeTok{diffan.obj =}\NormalTok{ dds\_list, }\AttributeTok{p=}\FloatTok{0.05}\NormalTok{, }\AttributeTok{q=}\FloatTok{0.1}\NormalTok{, }\AttributeTok{f=}\DecValTok{1}\NormalTok{,}
                            \AttributeTok{mark=}\NormalTok{mark, }\AttributeTok{outdir =}\NormalTok{ outdirsub.rich)}
\FunctionTok{save.image}\NormalTok{(}\AttributeTok{file =} \FunctionTok{paste0}\NormalTok{(outdirsub, }\StringTok{"/1.diff.img.RDATA"}\NormalTok{))}
\end{Highlighting}
\end{Shaded}

\hypertarget{deg-valcano}{%
\section{火山图}\label{deg-valcano}}

需要修改的是以下几个值:
df\_valcano:文件读取时的文件路径
ffdr: FDR阈值
fpval: PValue阈值
flogfc: logFC阈值
filterc: 火山图展示模式(p,fdr均考虑模式, 仅考虑p值模式,仅考虑fdr值模式,默认为第一种''fppadj'')
label\_gene: 展示基因列表
不清楚建议默认:fdr=0.2, pval=0.05, p-fdr均考虑模式。仅修改\texttt{label\_gene:\ 展示基因列表}即可。

\begin{Shaded}
\begin{Highlighting}[]
\CommentTok{\# rm(list = ls());gc()}
\CommentTok{\#library(ggpubr);library(ggrepel);library(ggsci);library(scales)}
\FunctionTok{library}\NormalTok{(tidyverse);}\FunctionTok{library}\NormalTok{(dplyr);}\FunctionTok{library}\NormalTok{(pheatmap);}\FunctionTok{library}\NormalTok{(RColorBrewer)}
\CommentTok{\# 导入火山图需要的差异分析后的基因全部表格(我们也称这个对象为resdf,}
\CommentTok{\#  resdf文件涵盖差异分析的所有结果信息,可以做后续所有基于差异分析或者基因列表}
\CommentTok{\#  的所有分析,如果后续分析时用到了不同的数据,请按这个resdf的格式改数据,主要}
\CommentTok{\#  就是把数据的列名改成和resdf的列名相同即可用此包的函数分析)}
\NormalTok{df\_valcano }\OtherTok{\textless{}{-}}\NormalTok{ readxl}\SpecialCharTok{::}\FunctionTok{read\_xlsx}\NormalTok{(}\StringTok{"./result/1.diff/rich/DIFF.an\_SHUANG{-}CONTROL.xlsx"}\NormalTok{,}
                                \AttributeTok{sheetIndex =} \DecValTok{1}\NormalTok{)}
\CommentTok{\# saveRDS(df\_valcano, file = "./result/1.diff/rich/DIFF.an\_SHUANG{-}CONTROL.rds")}
\CommentTok{\# df\_valcano \textless{}{-} readRDS("./result/1.diff/rich/DIFF.an\_SHUANG{-}CONTROL.rds")}
\FunctionTok{names}\NormalTok{(df\_valcano) }\CommentTok{\# 对应的列名必须为Gene, log2FC, PValue, FDR}
\CommentTok{\# 阈值设定}
\NormalTok{ffdr }\OtherTok{\textless{}{-}} \FloatTok{0.2}
\NormalTok{fpval }\OtherTok{\textless{}{-}} \FloatTok{0.05}
\NormalTok{flogfc }\OtherTok{\textless{}{-}} \DecValTok{1}
\CommentTok{\# 模式设定}
\CommentTok{\# pvalue, padj均考虑模式 ("fpadj":仅考虑fdr值模式, "other": 仅考虑p值模式)}
\NormalTok{filterc }\OtherTok{\textless{}{-}} \StringTok{"fppadj"} 
\CommentTok{\# 火山图数据预处理}
\NormalTok{pic\_data }\OtherTok{\textless{}{-}} \FunctionTok{DEG\_prepareVolcano}\NormalTok{(}\AttributeTok{df\_valcano =}\NormalTok{ df\_valcano, }\AttributeTok{filterc =}\NormalTok{ filterc)}

\CommentTok{\# 设定需要标记的marker gene}
\NormalTok{label\_gene }\OtherTok{\textless{}{-}} \FunctionTok{c}\NormalTok{(}\StringTok{\textquotesingle{}TFRC\textquotesingle{}}\NormalTok{, }\StringTok{\textquotesingle{}ACSL1\textquotesingle{}}\NormalTok{, }\StringTok{\textquotesingle{}LPCAT3\textquotesingle{}}\NormalTok{, }\StringTok{\textquotesingle{}PCBP1\textquotesingle{}}\NormalTok{, }\StringTok{\textquotesingle{}FTH1\textquotesingle{}}\NormalTok{, }\StringTok{\textquotesingle{}SLC11A2\textquotesingle{}}\NormalTok{,}
                \StringTok{\textquotesingle{}SLC39A8\textquotesingle{}}\NormalTok{, }\StringTok{\textquotesingle{}SAT1\textquotesingle{}}\NormalTok{, }\StringTok{\textquotesingle{}FTL\textquotesingle{}}\NormalTok{, }\StringTok{\textquotesingle{}GSS\textquotesingle{}}\NormalTok{)}
\CommentTok{\# 查看想展示的基因在不在差异分析总表中}
\CommentTok{\# label\_gene \%in\% df\_valcano$Gene \%\textgreater{}\% all}
\CommentTok{\# pic\_data \%\textgreater{}\% filter(Row.names \%in\% label\_gene)}

\CommentTok{\# 火山图 无标记}
\FunctionTok{DEGplot\_Volcano}\NormalTok{(}\AttributeTok{result =}\NormalTok{ pic\_data, }\AttributeTok{logFC =}\NormalTok{ flogfc,}
               \AttributeTok{adj\_P =}\NormalTok{ ffdr, }\AttributeTok{label\_geneset =} \ConstantTok{NULL}\NormalTok{)}
\CommentTok{\# 保存图片}
\CommentTok{\# ggsave("./valcano.pdf", width = 7, height = 7)}
\CommentTok{\# 火山图 有标记}
\FunctionTok{DEGplot\_Volcano}\NormalTok{(}\AttributeTok{result =}\NormalTok{ pic\_data, }\AttributeTok{logFC =}\NormalTok{ flogfc, }\CommentTok{\# log2(2)}
               \AttributeTok{adj\_P =}\NormalTok{ ffdr, }\AttributeTok{label\_geneset =}\NormalTok{ label\_gene) }\SpecialCharTok{\%\textgreater{}\%}
  \FunctionTok{ggplotGrob}\NormalTok{() }\SpecialCharTok{\%\textgreater{}\%}\NormalTok{ cowplot}\SpecialCharTok{::}\FunctionTok{plot\_grid}\NormalTok{()}
\CommentTok{\# 保存图片}
\CommentTok{\# ggsave("./result/1.diff/valcano.mark.gene.pdf", width = 7, height = 7)}
\end{Highlighting}
\end{Shaded}

\hypertarget{ux70edux56fe}{%
\section{热图}\label{ux70edux56fe}}

On the way \ldots{}

\hypertarget{enrich}{%
\chapter{富集分析}\label{enrich}}

\hypertarget{enrich-auto}{%
\section{GO \& KEGG 富集分析}\label{enrich-auto}}

\hypertarget{ux4e00ux952eux811aux672cux6279ux91cfux5904ux7406}{%
\subsection{一键脚本(批量处理)}\label{ux4e00ux952eux811aux672cux6279ux91cfux5904ux7406}}

这是一个一键脚本,请新建一个单独的文件写这段脚本,然后按这个脚本的顶部注释修改\texttt{resdf\ outd\ fc.list处即可},运行即可批量出不同FC的富集分析结果。

\begin{Shaded}
\begin{Highlighting}[]
\CommentTok{\# 此脚本为GO、KEGG分析(需要一个输入文件即可,为差异分析流程后的resdf文件)}
\CommentTok{\# 即为第一步(或1脚本)的结果的一个结果文件(DIFF\_an\_***.xlsx)}
\CommentTok{\# 即为resdf文件,此文件是差异分析后的总表}
\CommentTok{\# 注意如果采用了其他的分析方法得到差异分析后表,运行这个脚本时可能需要更改列名}
\CommentTok{\# 即我们的resdf对象的列名为Gene, log2FC,PValue,FDR,需要与这些个列名保持一致。}
\CommentTok{\# 此脚本中的需要修改的位于 /// *** /// 行中,另外还有一个LZ::setproxy()行,}
\CommentTok{\#   如果没有代理工具,或者代理工具不支持http代理,或者端口不通,请不要运行。}
\FunctionTok{rm}\NormalTok{(}\AttributeTok{list =} \FunctionTok{ls}\NormalTok{());}\FunctionTok{gc}\NormalTok{() }\CommentTok{\# 清空所有对象,慎用,必要时用}
\FunctionTok{suppressMessages}\NormalTok{(\{ }\FunctionTok{suppressWarnings}\NormalTok{(\{}
  \FunctionTok{library}\NormalTok{(LZ)}
  \FunctionTok{library}\NormalTok{(tidyverse);}\FunctionTok{library}\NormalTok{(data.table)}
  \FunctionTok{library}\NormalTok{(clusterProfiler);}\FunctionTok{library}\NormalTok{(enrichplot)}
  \FunctionTok{library}\NormalTok{(topGO);}\FunctionTok{library}\NormalTok{(Rgraphviz)}
  \FunctionTok{library}\NormalTok{(RColorBrewer);}\FunctionTok{library}\NormalTok{(ggsci);}\FunctionTok{library}\NormalTok{(pheatmap)}
  \FunctionTok{library}\NormalTok{(xlsx);}\FunctionTok{library}\NormalTok{(readxl)}
\NormalTok{\}) \})}
\CommentTok{\# 若无代理工具,切勿运行 }
\CommentTok{\# LZ::setproxy() \# 高危!!!新手不要运行此行,会使当前窗口断网!!!}
\CommentTok{\# Sys.getenv(\textquotesingle{}http\_proxy\textquotesingle{}) Sys.setenv(\textquotesingle{}http\_proxy\textquotesingle{}=\textquotesingle{}\textquotesingle{}) Sys.setenv(\textquotesingle{}https\_proxy\textquotesingle{}=\textquotesingle{}\textquotesingle{})}



\CommentTok{\# /// 1,2,3均需按实际改写}
\CommentTok{\# 读取数据  resdf存放目录}
\NormalTok{resdf }\OtherTok{\textless{}{-}}\NormalTok{ readxl}\SpecialCharTok{::}\FunctionTok{read\_xlsx}\NormalTok{(}\StringTok{"result/rnaseqOR{-}NC/rich/DIFF.an\_OR{-}NC.xlsx"}\NormalTok{,}
                  \AttributeTok{sheet =} \DecValTok{1}\NormalTok{) }\SpecialCharTok{\%\textgreater{}\%} \FunctionTok{as.data.frame}\NormalTok{()}
\CommentTok{\# 此处可能需要插入修改列名的代码,需要为标准的resdf格式}
\CommentTok{\#   标准resdf格式,用列名Gene, log2FC, PValue, FDR来表示gene, log2fc, p, q/fdr}

\CommentTok{\# 输出目录}
\NormalTok{outd }\OtherTok{=} \StringTok{"result/rnaseqOR{-}NC/rich"} 
\CommentTok{\# logFC阈值, 多个阈值的话,写成fc.list \textless{}{-} list(\textquotesingle{}1.2\textquotesingle{} = log2(1.2), \textquotesingle{}2\textquotesingle{} = log2(2))}
\CommentTok{\# 注意!!!!!!:括号里log2(2)的2,和引号里\textquotesingle{}2\textquotesingle{}的2都要需同步要改。!!!}
\CommentTok{\# 否则可能会覆盖结果}
\NormalTok{fc.list }\OtherTok{\textless{}{-}} \FunctionTok{list}\NormalTok{(}\StringTok{\textquotesingle{}2\textquotesingle{}}\OtherTok{=}\FunctionTok{log2}\NormalTok{(}\DecValTok{2}\NormalTok{))}
\CommentTok{\# 设置物种为人类(如是人类则不需要更改)}
\NormalTok{GO\_database }\OtherTok{\textless{}{-}} \StringTok{\textquotesingle{}org.Hs.eg.db\textquotesingle{}} \CommentTok{\# keytypes(org.Hs.eg.db)}
\NormalTok{KEGG\_database }\OtherTok{\textless{}{-}} \StringTok{\textquotesingle{}hsa\textquotesingle{}} 
\CommentTok{\# ///}


\CommentTok{\# 预处理数据符合GOKEGG分析的要求}
\CommentTok{\# \# 不同fc条件下的GOgenelist list(ALL, UP, DOWN)}
\NormalTok{gogenelist }\OtherTok{\textless{}{-}} \FunctionTok{lapply}\NormalTok{(fc.list, }\ControlFlowTok{function}\NormalTok{(x) }\FunctionTok{DEG\_prepareGOglist}\NormalTok{(resdf, }\AttributeTok{logfc =}\NormalTok{ x))}
\CommentTok{\# gogenelist \%\textgreater{}\% length()}
\CommentTok{\#}
\CommentTok{\# 对logFC迭代,每个FC新建一个目录,用来存upgene, downgene, allgene的GO结果}
\NormalTok{go }\OtherTok{\textless{}{-}} \FunctionTok{DEG\_runGO}\NormalTok{(}\AttributeTok{outdir =}\NormalTok{ outd, }\AttributeTok{genelist =}\NormalTok{ gogenelist)}
\CommentTok{\# 对logFC迭代,每个FC新建一个目录,用来存upgene, downgene, allgene的KEGG结果}
\NormalTok{kegg }\OtherTok{\textless{}{-}} \FunctionTok{DEG\_runKEGG}\NormalTok{(}\AttributeTok{outdir =}\NormalTok{ outd, }\AttributeTok{genelist =}\NormalTok{ gogenelist)}
\end{Highlighting}
\end{Shaded}

\hypertarget{enrich-simple}{%
\subsection{简易GO,KEGG一次分析}\label{enrich-simple}}

如果已经得到了差异基因列表,且无需批量分析,可以进行这个简易分析。
数据格式: \texttt{head(genelist.lh)}
{[}1{]} ``AARS1'' ``AATF'' ``ABCB7'' ``ABCE1'' ``ABHD11'' ``ABHD12''

\begin{Shaded}
\begin{Highlighting}[]
\CommentTok{\# 简易GO,KEGG一次分析(即:已经得到了差异基因列表)}
\CommentTok{\# LZ::setproxy() \# 代理设置,新手别碰,会断网}
\CommentTok{\# 差异基因列表}
\NormalTok{genelist.lh }\OtherTok{\textless{}{-}}\NormalTok{ pic.list}\SpecialCharTok{$}\NormalTok{sig.data}\SpecialCharTok{$}\NormalTok{Gene }
\CommentTok{\# 转换ID}
\NormalTok{gene\_df }\OtherTok{\textless{}{-}} \FunctionTok{bitr}\NormalTok{(genelist.lh, }\AttributeTok{fromType =} \StringTok{"SYMBOL"}\NormalTok{, }\AttributeTok{toType =} \FunctionTok{c}\NormalTok{(}\StringTok{"ENTREZID"}\NormalTok{, }\StringTok{"UNIPROT"}\NormalTok{), }
                \AttributeTok{OrgDb =} \StringTok{\textquotesingle{}org.Hs.eg.db\textquotesingle{}}\NormalTok{)}
\CommentTok{\# GO分析}
\NormalTok{go.lh }\OtherTok{\textless{}{-}} \FunctionTok{DEG\_GO}\NormalTok{(gene\_df, }\AttributeTok{orgdb =} \StringTok{"org.Hs.eg.db"}\NormalTok{, }\AttributeTok{sigNodes =} \DecValTok{20}\NormalTok{, }
                \AttributeTok{resultdir=}\StringTok{"./result/proteinOR{-}NC"}\NormalTok{, }\AttributeTok{filemark =} \StringTok{"p1.5\_g\_2"}\NormalTok{)}
\NormalTok{go.lhdf }\OtherTok{\textless{}{-}} \FunctionTok{sapply}\NormalTok{(go.lh, }\ControlFlowTok{function}\NormalTok{(x) x}\SpecialCharTok{@}\NormalTok{result, }\AttributeTok{simplify =}\NormalTok{ T)}
\FunctionTok{write\_xlsx}\NormalTok{(go.lhdf, }\AttributeTok{path =} \StringTok{"./result/proteinOR{-}NC/lh\_go.all.xlsx"}\NormalTok{)}
\CommentTok{\# KEGG分析}
\NormalTok{kegg.lh }\OtherTok{\textless{}{-}} \FunctionTok{DEG\_KEGG}\NormalTok{(gene\_df)}
\FunctionTok{write\_xlsx}\NormalTok{(kegg.lh}\SpecialCharTok{$}\NormalTok{pSigDF, }\AttributeTok{path =} \StringTok{"./result/proteinOR{-}NC/lh\_kegg.all.xlsx"}\NormalTok{)}
\end{Highlighting}
\end{Shaded}

\hypertarget{dotplot}{%
\subsection{GO、KEGG分析结果可视化 \{\#enrich-visual\}}\label{dotplot}}

\begin{Shaded}
\begin{Highlighting}[]
\CommentTok{\# dotplot go}
\CommentTok{\# 读取go分析保存的表格}
\NormalTok{dotData }\OtherTok{\textless{}{-}}\NormalTok{ readxl}\SpecialCharTok{::}\FunctionTok{read\_xlsx}\NormalTok{(}\StringTok{"./result/proteinOR{-}NC/lh\_go.all.xlsx"}\NormalTok{, }\AttributeTok{sheet =} \DecValTok{1}\NormalTok{)}
\CommentTok{\# 筛选数据(按需配合其他筛选)}
\NormalTok{dotData }\OtherTok{\textless{}{-}} \FunctionTok{DEGp\_prepareDotplot}\NormalTok{(dotData, }\AttributeTok{head =} \DecValTok{30}\NormalTok{, }\AttributeTok{delete =} \ConstantTok{NULL}\NormalTok{)}
\NormalTok{pic.dot }\OtherTok{\textless{}{-}} \FunctionTok{DEGp\_Dotplot}\NormalTok{(dotData, }\AttributeTok{title =} \StringTok{\textquotesingle{}TOP of GO\textquotesingle{}}\NormalTok{, }
                        \AttributeTok{resultdir =} \StringTok{"./result/proteinOR{-}NC"}\NormalTok{, }\AttributeTok{filemark =} \StringTok{\textquotesingle{}GO\_top\textquotesingle{}}\NormalTok{, }
                        \AttributeTok{pic.save =}\NormalTok{ T)}

\CommentTok{\# dotplot kegg}
\CommentTok{\# 读取kegg分析保存的表格}
\NormalTok{dotDatak }\OtherTok{\textless{}{-}}\NormalTok{ readxl}\SpecialCharTok{::}\FunctionTok{read\_xlsx}\NormalTok{(}\StringTok{"./result/proteinOR{-}NC/lh\_kegg.all.xlsx"}\NormalTok{, }\AttributeTok{sheet =} \DecValTok{1}\NormalTok{)}
\CommentTok{\# 筛选数据(按需配合其他筛选)}
\NormalTok{dotDataK }\OtherTok{\textless{}{-}} \FunctionTok{DEGp\_prepareDotplot}\NormalTok{(dotDatak, }\AttributeTok{head =} \DecValTok{30}\NormalTok{, }\AttributeTok{delete =} \ConstantTok{NULL}\NormalTok{)}
\NormalTok{pic.dotk }\OtherTok{\textless{}{-}} \FunctionTok{DEGp\_Dotplot}\NormalTok{(dotDataK, }\AttributeTok{title =} \StringTok{\textquotesingle{}TOP of KEGGpathway\textquotesingle{}}\NormalTok{, }
                         \AttributeTok{resultdir =} \StringTok{"./result/proteinOR{-}NC"}\NormalTok{, }\AttributeTok{filemark =} \StringTok{\textquotesingle{}KEGG\_top\textquotesingle{}}\NormalTok{, }
                         \AttributeTok{pic.save =}\NormalTok{ F)}

\CommentTok{\# 组合图}
\NormalTok{gh }\OtherTok{\textless{}{-}} \FunctionTok{ggplotGrob}\NormalTok{(pic.dot)}
\NormalTok{gd }\OtherTok{\textless{}{-}} \FunctionTok{ggplotGrob}\NormalTok{(pic.dotk)}
\NormalTok{cowplot}\SpecialCharTok{::}\FunctionTok{plot\_grid}\NormalTok{(gh, gd, }\AttributeTok{rel\_widths =} \FunctionTok{c}\NormalTok{(}\DecValTok{1}\NormalTok{, }\FloatTok{1.25}\NormalTok{))}
\FunctionTok{ggsave}\NormalTok{(}\FunctionTok{paste0}\NormalTok{(dir\_out, }\StringTok{"/GO\_KEGG\_top.pdf"}\NormalTok{), }\AttributeTok{width =} \DecValTok{16}\NormalTok{, }\AttributeTok{height =} \DecValTok{10}\NormalTok{)}
\end{Highlighting}
\end{Shaded}

\hypertarget{enrich-gsea}{%
\section{GSEA分析}\label{enrich-gsea}}

\hypertarget{r-gsea}{%
\subsection{R GSEA}\label{r-gsea}}

On the way \ldots{}

\hypertarget{gseaux5b98ux65b9ux8f6fux4ef6}{%
\subsection{GSEA官方软件}\label{gseaux5b98ux65b9ux8f6fux4ef6}}

On the way \ldots{}

\hypertarget{visual}{%
\chapter{差异及富集分析可视化专题}\label{visual}}

On the way \ldots{}

\hypertarget{rnaseq-rsubread}{%
\chapter{RNAseq上游流程}\label{rnaseq-rsubread}}

On the way \ldots{}

\hypertarget{multi-omics}{%
\chapter{多组学}\label{multi-omics}}

On the way

\hypertarget{cuttag}{%
\chapter{CUT\&TAG}\label{cuttag}}

On the way \ldots{}

\hypertarget{scRNA}{%
\chapter{单细胞分析}\label{scRNA}}

On the way \ldots{}

\hypertarget{spatial}{%
\chapter{空间转录组}\label{spatial}}

On the way \ldots{}

  \bibliography{book.bib,packages.bib}

\end{document}
